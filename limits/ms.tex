\documentclass[onecolumn,galley]{mn2e}
%\documentclass[twocolumn,galley]{mn2e}
%\documentclass[onecolumn,galley,draft]{mn2e}
%\usepackage{myaasmacros}
\usepackage{mathrsfs}
\usepackage{amsmath}
\usepackage{graphicx}

\newcommand{\Glass}{{\sc Glass}}
\newcommand{\PixeLens}{{\sc PixeLens}}
\newcommand{\Rmap}{\ensuremath{R_\mathrm{map}}}
\newcommand{\Rpix}{\ensuremath{R_\mathrm{pix}}}
\newcommand{\M}{\ensuremath{\mathscr{M}}}
\newcommand{\E}{\ensuremath{\mathscr{E}}}
\newcommand{\eps}{\ensuremath{\varepsilon}}

\newcommand{\Eavg}{\ensuremath{\langle \E \rangle}}

\newcommand{\kpc}{\ensuremath{\mathrm{kpc}}}
\newcommand{\Msun}{\ensuremath{\mathrm{M}_\odot}}
\newcommand{\tabref}[1] {Table~\ref{#1}}
\newcommand{\figref}[1] {Figure~\ref{#1}}
\newcommand{\eqnref}[1] {Eq.~(\ref{#1})}
\newcommand{\secref}[1] {Section~\ref{#1}}
\newcommand{\appref}[1] {Appendix~\ref{#1}}
\newcommand{\e}[1]{\ensuremath{\times 10^{#1}}}
\renewcommand{\vec}[1]{\ensuremath{\boldsymbol{#1}}}

% From aastex.cls
\newcommand\plotone[1]{%
 \centering
 \leavevmode
 \includegraphics[width={\columnwidth}]{#1}%
}%
\newcommand\plottwo[2]{{%
 \centering
 \leavevmode
 \columnwidth=.45\columnwidth
 \includegraphics[width={\columnwidth}]{#1}%
 \hfil
 \includegraphics[width={\columnwidth}]{#2}%
}}%
\newcommand\plotthree[3]{{%
 \centering
 \leavevmode
 \columnwidth=.30\columnwidth
 \includegraphics[width={\columnwidth}]{#1}%
 \hfil
 \includegraphics[width={\columnwidth}]{#2}%
 \hfil
 \includegraphics[width={\columnwidth}]{#3}%
}}%


%\title{Limits on constraining galaxy profiles with current and future strong gravitational lensing}
\title[GLASS]{Lens Recovery with GLASS: An Extensible Free-Form Gravitational Lens Modeling Framework}
\author{%
Jonathan P. Coles 
\and 
Justin I. Read
\and 
Prasenjit Saha 
}

\begin{document}
\maketitle

%\tableofcontents

\begin{abstract}
We show the effectiveness of reconstructing the mass distribution of four mock galaxies using
a range of strong gravitational lensing assumptions. In particular, we study
the affects of the lens morphology (double or quad) with and without a central
image, angular image distribution, and red shift contrast between two sources.
The lens modeling is performed using a new free-form framework called \Glass.
We conclude that \dots
%1) Time delays are unnecessary with strong red shift contrast,
%2)



%We study the effectiveness of a variety of strong lensing scenarios to inform
%the reconstruction of four simulated triaxial halos with stellar and dark
%matter components. We conclude that 
%1) lenses with multiple sources and high redshift contrast perform very well
%2) lenses where the images are already probing the stellar dominated region 
%   of a galaxy can be constrained by an estimate of the stellar mass and
%   suggest the fraction of dark matter
%3) if velocity dispersion measurements are to be used then a sufficiently large aperture 
%   is necessary
%4) lensing can be used to infer the velocity anisotropy.
%We also introduce GLASS, an extensible framework for lens modeling and analysis.

%We attempt to reconstruct the mass profile and distribution of a dark and
%stellar matter triaxial halo using a variety of strong lens configurations,
%stellar mass estimates, and stellar velocity dispersions. The lens
%configurations test quads with small to large image separations (radial
%sampling), extended source information ranging from point images to Einstein
%rings (angular sampling), and multiple sources located at the same and
%differing redshifts (redshift contrast). 

%We conclude that redshift contrast can significantly constrain the mass
%profiles.  Stellar mass estimates can also lead to better reconstructions by
%placing lower bounds on the mass distribution in the inner regions of galaxies.
%Velocity dispersions alone are unlikely to usefully constrain the profile
%because the profile is highly sensitive to the dispersion measurement.  
%Combined with other techniques, however, velocity dispersions can help to
%constrain the range of possible reconstructed solutions. We also introduce
%GLASS, an extensible framework for lens modeling and analysis.  
\end{abstract}

\section{Introduction}

To date only about 400 strong lenses are known, but within the next ten years
new surveys will discover thousands of gravitational lenses. The SKA estimates
that after its start in 2020 about 10,000 lenses will be found. The LSST
expects to find about 5,000 multiply-imaged galaxies and about 150 AGNs or
quasar lenses after 2022.

A typical lens, where only the measured observables are the image positions,
can been used to estimate the total mass of a galaxy including parts of the
dark matter halo. The enclosed mass inside the Einstein radius is simple to
calculate and quite robust given the observed image positions. The exact mass
distribution is more difficult to determine, however, as there are typically
not enough constraints on the mass profile. The known degeneracies have been
discussed at length in the literature. At the very least, one would like to
have a measure of the enclosed mass at two distinct radii to be able
to properly distinguish between analytic mass models. 
%This would break the steepness degeneracy, also known as the mass sheet degeneracy.

Lensing is not the only technique that can measure enclosed mass. The enclosed
mass can also be estimated from stellar velocity dispersions. If the radius
from this measurement is sufficiently different from the Einstein radius, then
combining this measurement with lensing provides the necessary constraint.
Another solution is to find a galaxy that is lensing two sources. The images
from each source provide an enclosed mass. The extent to which the radii
differ is dependent upon the redshifts of the two sources: the larger the
redshift difference the larger the radial separation.

Lensing is known to be sensitive to cosmology and, in principle, can be used to
estimate such quantities as the Hubble constant $H_0$ when the lens has
measurable time delays. However, due to strong degeneracies such as the
steepness degeneracy there is often little one can learn from a single lens. 
In this case, one is free to change the mass and $H_0$ while keeping
all observables fixed. A single lens can say almost nothing about the actual
value of $H_0$. However, it has been shown that combining modeling results
of several strong lenses with time delays can significantly reduce the allowed
values and provides a measure to within 10 percent \cite{}.

If one assumes a Hubble constant $H_0$ and time delays have been observed for a
lens this will break the steepness degeneracy but will not help in
constraining the mass profile at two radii.  One can choose to postpone this
problem and simply fit models to the given data.  The earliest examples were
singular isothermal sphere (SIS) or elliptical (SIE) models but have since
evolved to multi-component models include separate models for the stars and
dark matter. Other methods that do not ignore this problem use genetic
algorithms or linear programming to explore the large space of solutions.

Despite such difficulties lensing is a direct probe of the gravitating mass in
a halo and there are a number of open problems in galaxy formation that lensing
can help address. For instance, the debate remains as to what form the inner
slope of the density profile takes. Numerical simulations of dark matter
suggest a cusp, while observations from lensing and stellar dynamics suggest
cores. This may, however, be only a matter of scale, as both theory and
observations agree that clusters have cuspy profiles. 

Lensing may also be useful in measuring possible adiabatic contraction of the 
dark matter halo due to the central stars in a galaxy. For this, one needs
high resolution mass models of the stellar profile combined with accurate models
of the total mass. Indeed, with such models, one may also improve the statistics
of the dark matter fraction in a galaxies over many different mass scales.

These open problems suggest what kind of lensing data will be required in the
future.  Future surveys will provide many lenses but time delays will require a
long follow up. Stellar mass maps, however, take a known, fixed observation
time and can be performed for many lenses. It will be important not only to
measure the stellar velocity dispersion at an arbitrary radius, but to ensure
that the measurement is taken at a radius different to the Einstein radius.
For the radio surveys like SKA it will be important to detect the central image
of the lens. This can provide mass information all the way to the center of the
galaxy.

In this paper we perform a parameter study using four mock galaxies that differ
in their dark matter and stellar profiles. We consider double and quad lensing
morphologies, time delays, the presence of a central image, red shift contrast
between multiple sources, and image configuration.  We also consider the
effects of adding stellar mass models as lower bound constraints on the mass
distribution.  The modeling and analysis is performed using a new free form
lens modeling framework called \Glass. 

We first review the basic lensing theory we will need and then describe how
\Glass is used to generate lensing models. The mock data and how it was
generated is described in \secref{}. In section{} we discuss the lensing
configurations in the parameter study and present some example output. Our
finals results are presented in \secref{}.


%\subsection{Future} %-----------------------------------------------------------
    %\begin{itemize}
    %\item New satellites will find many new lenses.
    %\item LSST, DES, PanSTARRS?, SKA
    %\item What is the most important data to have?
    %\end{itemize}


%\subsection{Present} %----------------------------------------------------------
%    \begin{itemize}
%    \item SLACS, CASTLES, CosmoGrail, CLASS, CFHTLS, SNAP?
%    \end{itemize}

%\subsection{Why do we need accurate galaxy models?} %---------------------------
    %\begin{itemize}
    %\item Galaxy formation
    %\item Cusp/Core problem
    %\item Universal halos?
    %\item Adiabatic contraction
    %\item DM fraction
    %\end{itemize}


%\subsection{Past work on constraining profiles.} %------------------------------
    %\begin{itemize}
    %\item Much of the early literature used simple SIS and SIE parametric models. Understandable.
    %\item Bayesian model fitting
    %\item Free-form modeling
    %\item Keeton has a good review.
    %\end{itemize}

%\subsection{What are the problems?} %------------------------------
    %\begin{itemize}
    %\item Very little data per lens.
    %\item Several well-known degeneracies.
    %\item Velocity dispersions: aperture too small
    %\end{itemize}

%\subsection{Goals of the paper} %------------------------------
%    \begin{itemize}
%    \item Perform a parameter study over lens morphologies and quantify what they can tell us about profiles.
%    \end{itemize}

%%%%%%%%%%%%%%%%%%%%%%%%%%%%%%%%%%%%%%%%%%%%%%%%%%%%%%%%%%%%%%%%%%%%%%%%%%%%%%%%
%%%%%%%%%%%%%%%%%%%%%%%%%%%%%%%%%%%%%%%%%%%%%%%%%%%%%%%%%%%%%%%%%%%%%%%%%%%%%%%%
%%%%%%%%%%%%%%%%%%%%%%%%%%%%%%%%%%%%%%%%%%%%%%%%%%%%%%%%%%%%%%%%%%%%%%%%%%%%%%%%
%%%%%%%%%%%%%%%%%%%%%%%%%%%%%%%%%%%%%%%%%%%%%%%%%%%%%%%%%%%%%%%%%%%%%%%%%%%%%%%%

\section{Theoretical Background}

\subsection{Lensing}

The lensing equation
%
\begin{equation}
\vec\beta = \vec\theta - \frac{D_{LS}}{D_S}\vec\alpha(\vec\theta)
\label{lensing equation}
\end{equation}
%
maps an observed image position $\vec\theta$ to a source position $\vec\beta$
via the correction factor 
%
\begin{equation}
\vec\alpha(\vec\theta) = \frac{4GD_L}{c^2} \int \Sigma(\vec\theta')\frac{(\vec\theta - \vec\theta')}{\ |\vec\theta - \vec\theta'|^2}d^2\vec\theta'
\end{equation}
%
which accounts for the gravitational lensing effect of an intervening mass. For
our purposes we make use of the thin mass sheet approximation and assume that
the lensing mass is infinitely thin compared with the distance between the
source and the lens $D_{LS}$ and the lens and the observer $D_{L}$. These
distances are angular diameter distances where $D_* = (c/H_0)d_*$ and $d_*$ is
a dimensionless number that depends on cosmology.  The lens can then be thought
of as a projected surface density $\Sigma$ which diverts the path of a photon
instantaneously through the bending angle $\vec\alpha$.

Since we will later want to model the density distribution with a computer it will
be convenient to choose units that make the relevant quantities of order unity.
We therefore measure lengths in light years, time in years, positions in
arcseconds, and choose $c=1$ and $4\pi G = N^2$, where $N^2 \equiv 206,265$
arcsec/rad. The mass unit is then $11.988\ \Msun$. It will also be useful to 
define a proxy to the Hubble constant $\nu \equiv N^2 H_0$.

The arrival (or travel) time $t$ of a photon depends not only on the geometric
path it takes, but also on the general relativistic gravitational time dilation
due to the lens at a redshift of $z_L$.  In its complete form, with all units
and cosmology included we can write the arrival time as
%
\begin{equation}
N^2ct(\vec\theta) = (1+z_L)\frac{D_{L}D_{S}}{D_{LS}}\frac12 |\vec\theta - \vec\beta|^2 - (1+z_L)\frac{4GD_{L}^2}{c^2}\int \Sigma(\vec\theta') \ln |\vec\theta-\vec\theta'| d^2\vec\theta'
\label{full arrival time}
\end{equation}
%
The non-standard factor of $D_L^2$ in the second term comes from the fact that $\Sigma$
has units of \Msun/lyr$^2$. We can clean this up by writing down a dimensionless time
delay 
%
\begin{equation}
\tau = \left[(1+z_L)D_{L}\right]^{-1}ctN^2 = \left[ (1+z_L) \frac{c}{H_0}d_L\right]^{-1}ctN^2 = \left[ (1+z_L) d_L\right]^{-1}\nu t
\label{tau}
\end{equation}
%
in terms of our previous definitions. If we
further define a dimensionless density
%
\begin{equation}
\kappa_\infty = \frac{4\pi G}{c^2}\frac{c}{H_0}d_L\Sigma
              = \frac{d_L}{\nu}\Sigma
\end{equation}
%
and a lensing potential
%
\begin{equation}
\psi(\vec\theta) = \frac1\pi \int \kappa_\infty(\vec\theta') \ln|\vec\theta - \vec\theta'| d^2\vec\theta'\
\label{lensing potential}
\end{equation}
%
we can express \eqnref{full arrival time} very compactly as
%
\begin{equation}
\tau(\vec\theta) = \frac12 \xi |\vec\theta-\vec\beta|^2 - \psi(\vec\theta)
\label{arrival time}
\end{equation}
%
where $\xi \equiv D_{S}/D_{LS}$. We explicitly write $\kappa_\infty$ to remind ourselves
that there is no source distance factor involved. This will be useful later when we consider
multiple sources.

If we think of $\tau$ as a surface then images will be observed where ever
$\nabla \tau = 0$. In other words, images obey Fermat's principle and appear at
extremums of the surface. Note that $\nabla \tau$ is just \eqnref{lensing equation} or
\begin{equation}
  \xi \vec\beta = \xi \vec\theta - \frac 1\pi\int \kappa_\infty\frac{(\vec\theta - \vec\theta')}{\ |\vec\theta - \vec\theta'|^2}d^2\vec\theta'
\end{equation}

The only observables from \eqnref{arrival time} are the image positions, and
perhaps the stellar contribution to $\kappa_\infty$.  If the source object is
variable, however, then the observed light curves of each image will be
identical, but possibly shifted in time due to the different arrival times.
The time difference between identical portions of the light curves is called
the time delay $\Delta t$ between images.  Using \eqnref{tau} the time delay
$\Delta t_{21}$ between two images $\vec\theta_1$ and $\vec\theta_2$, where $\vec\theta_2$
arrives after $\vec\theta_1$, is $\Delta t_{21} = \Delta \tau_{21}(1+z_L)d_L /
\nu$. If $\beta$ and $\kappa_\infty$ were known exactly this would allow one
to precisely infer the Hubble constant. However, the source position is not
observable and the density distribution is highly degenerate.

\subsection{Degeneracies} %-----------------------------------------------------

\begin{itemize}
  \item Steepness
  \item Shape
  \item $H_0$
  \item Rings
  \item Cosmology
\end{itemize}

%
%The steepness degeneracy arbitrarily rescales the arrival time surface. Image
%positions are left alone but the density profile will become shallower or
%steeper. This is particularly bad for inferring the Hubble constant, as the
%lens profile will tend to become steeper for larger values of $H_0$.
%
%Other degeneracies include ring degeneracies, where one is allowed to build
%up a density distribution as a series of mass rings; the presence of substructure;
%and the effects of merging galaxies. 
%
%Shape degeneracy---a twisting in the density.

%%%%%%%%%%%%%%%%%%%%%%%%%%%%%%%%%%%%%%%%%%%%%%%%%%%%%%%%%%%%%%%%%%%%%%%%%%%%%%%%
%%%%%%%%%%%%%%%%%%%%%%%%%%%%%%%%%%%%%%%%%%%%%%%%%%%%%%%%%%%%%%%%%%%%%%%%%%%%%%%%
%%%%%%%%%%%%%%%%%%%%%%%%%%%%%%%%%%%%%%%%%%%%%%%%%%%%%%%%%%%%%%%%%%%%%%%%%%%%%%%%
%%%%%%%%%%%%%%%%%%%%%%%%%%%%%%%%%%%%%%%%%%%%%%%%%%%%%%%%%%%%%%%%%%%%%%%%%%%%%%%%
\section{Numerical Methods} 

\subsection{A new lens modeling framework: \Glass}

\Glass\ is an evolution of the concepts employed by the free form modeling tool
\PixeLens\ but completely rewritten in Python and C. The key improvements are:
%
\begin{enumerate}
  \setcounter{enumi}{0}
  \item A modular framework allows new priors to be added and modified easily.
\end{enumerate}
%
Each prior is a simple function that adds linear constraints that operate on either
a single lens object or the entire ensemble of objects. \Glass\ comes with a number
of useful priors, but a user can write their own in the input file.
%
\begin{enumerate}
  \setcounter{enumi}{1}
  \item The basis functions approximating a model can be changed. 
\end{enumerate}
%
\Glass\ currently describes the lens mass as a collection of pixels, but the code
has been designed to support alternative methods. In particular, it is planned
to develop a module using Bessel functions. This will require a new set of 
priors that operate on these functions.
%
\begin{enumerate}
  \setcounter{enumi}{2}
  \item Non-linear constraints can be imposed in an automated post-processing step. 
\end{enumerate}
%
Once \Glass\ has generated an ensemble of models given the linear constraints, any number
of post processing functions can be applied. Not only can these functions be used to
derive new quantities from the mass models, they can also be used as a filter to 
accept or reject a model based on some non-linear constraint. The plotting functions
within \Glass\ will correctly display models that have been accepted or rejected.
%
\begin{enumerate}
  \setcounter{enumi}{3}
  \item The central region can be have a higher resolution to capture steep models. 
\end{enumerate}
%
By default, the mass distribution of the lens is described by uniform grid. However,
in the central region of a lensing galaxy where the mass profile may rise steeply,
a higher resolution can be specified that only applies to the inner pixels. These
pixels are subdivided into a smaller grid.
%
\begin{enumerate}
  \setcounter{enumi}{4}
  \item Stellar density can be used as an additional constraint.  
\end{enumerate}
%
The mass in inner regions of galaxies is often dominated by the stellar component
which one can estimate using standard mass-to-light models. This data can be added
to the potential as described later in \secref{stellar mass}. By using the stellar
mass one can place a lower bound on the mass and help constrain the inner most
mass profile.
%
\begin{enumerate}
  \setcounter{enumi}{5}
  \item Point or extended mass objects can be placed in the field.
\end{enumerate}
%
\PixeLens only allows for a shear term to be added to the potential (as shown
later in \eqnref{shear}) to account for mass external to model region. This
is useful to capture the gross effects of a distant neighbour. \Glass\ also
has a shear term but additional allows for additional analytic potentials to
be included. This can be used to model substructure or multiple neighbors close
to the main lens. The substructure may have only a small effect if the lens is
a single galaxy, but if the lens is group then a potential can be added for
each of the galaxies.
%
\begin{enumerate}
  \setcounter{enumi}{5}
  \item A new uniform sampling algorithm for high dimensional spaces.
\end{enumerate}
%
At the heart of \Glass\ lies a new algorithm for sampling the high dimensional
linear space that represents the modeling solution space. This algorithm was
described and tested in \cite{}. \Glass\ is also multi-threaded allowing it to
run efficiently on many-cored machines.  The software is freely available for
download. One interesting aspect is that the input files are themselves Python
programs. This allows a large amount of sophistication in setting up a lens at
runtime and also allows \Glass\ to be used as an external library from another
user written program.

\subsection{Discrete models}

For this paper we will restrict ourselves to using the pixelated basis set that
has been employed in previous work and is implemented in \PixeLens. Here we
will briefly review the main ideas. The algorithm for generating models in
\Glass\ samples a convex polytope in a high dimensional space whose interior
points are solutions to the lensing equation and satisfy other physically
motivated linear priors (the sampling algorithm in \Glass\ is not the same as
employed in \PixeLens).  We therefore formulate all of our equations as
equations linear in the unknowns. We describe the density distribution $\kappa$
as a set of discrete grid cells or pixels $\kappa_i$ and rewrite the potential
(\eqnref{lensing potential}) as
%
\begin{equation}
  \psi(\vec\theta) = \sum_n \kappa_n Q_n(\vec\theta)
  \label{discrete potential}
\end{equation}
%
where the sum runs over all the pixels and $Q_n$ is the integral of the logarithm
over pixel $n$. The exact form for $Q$ is described in \appref{Q derivation}.
We can find the discretized lens equation by simply taking the gradient of the
above equations. 

The pixels only cover a finite circular area with physical radius $\Rmap$ and
pixel radius $\Rpix$ centered on the lensing galaxy. To account for any global
shearing outside this region from, e.g., a neighboring galaxy, we also add to
\eqnref{discrete potential} two shearing terms
%
\begin{equation}
\label{shear}
\gamma_1(\theta_x^2 - \theta_y^2) + 2\gamma_2\theta_x\theta_y\quad.
\end{equation}
%
We can continue adding terms to account for other potentials. For instance,
we may want to impose a base potential over the field, or add potentials
from the presence of other galaxies in the field. \Glass\
already includes potentials for a point mass or an exponential form, but custom
potentials are straightforward to add and can be included directly in the input file.
If the stellar density has been estimated we can use this as a lower bound
where the stellar potential is a known constant of the form \eqnref{discrete
potential}.

%
%\begin{equation}
  %\psi^*(\vec\theta) = \sum_n \kappa^*_n Q_n(\vec\theta)\quad.
%\end{equation}
%

Unfortunately, the lens equation and the arrival times alone are not enough to form a
closed volume in the solution space. We have not imposed any strong constraints
on the density even so much as to ensure that it is non-negative everywhere. We
must therefore provide additional linear constraints so that we only sample
physical models. The explicit implementation for these priors has been
explained in \cite{}, but can be summarized as
\begin{enumerate}
\item The density must be non-negative everywhere.
\item The density profile must have a slope everywhere $\le 0$. This prevents hollow rings.
\item The local gradient everywhere must point within $45^{\circ}$ of the center.
\item Density variations must be smooth.
\item Image parity is enforced.
\item The density is radially symmetric. (Optional)
\end{enumerate}

In the simplest form, a single model for a lens is a tuple $\M = (\vec\kappa,
\beta, \gamma_1, \gamma_2)$. A single model represents a single point in the
solution space polytope. Using the MCMC sampling strategy described in \cite{}
we uniformly sample this space. Collectively, the sampled models are referred
to as an ensemble $\E = \{\M_i\}$, where we usually generate $\sim$1000 models. We
will typically show the ensemble average $\Eavg$ in the plots to
follow.


%%%%%%%%%%%%%%%%%%%%%%%%%%%%%%%%%%%%%%%%%%%%%%%%%%%%%%%%%%%%%%%%%%%%%%%%%%%%%%%%
%%%%%%%%%%%%%%%%%%%%%%%%%%%%%%%%%%%%%%%%%%%%%%%%%%%%%%%%%%%%%%%%%%%%%%%%%%%%%%%%
%%%%%%%%%%%%%%%%%%%%%%%%%%%%%%%%%%%%%%%%%%%%%%%%%%%%%%%%%%%%%%%%%%%%%%%%%%%%%%%%
%%%%%%%%%%%%%%%%%%%%%%%%%%%%%%%%%%%%%%%%%%%%%%%%%%%%%%%%%%%%%%%%%%%%%%%%%%%%%%%%

\subsection{Raytracing}
\label{Raytracing}
\Glass\ can also determine the position of images and time delays from 
particle-based simulation output given a source position $\vec\beta$. This is
used to generate the lens configurations used in the parameter study.  The
particles are first projected onto a very high resolution grid representing the
lens plane. The centers $\vec\theta_i$ of each of the grid cells are mapped
back onto the source plane using \eqnref{lensing equation}. If the location on
the source plane $\vec\beta_i$ is within a user specified
$\eps_\mathrm{accept}$ of $\vec\beta$ then $\vec\theta_i$ is 
accepted and further refined using a root finding algorithm until the distance
to $\vec\beta$ is nearly zero. If multiple points converge to an
$\eps_\mathrm{root}$ of each other then only one point is taken.  Care must be
taken that the grid resolution is high enough that the resulting image position
error is below the equivalent observational error. Time delays are then
calculated in order from the arrival time at each image (\eqnref{tau}).

\section{Parameter study}

We now present a study of four mock galaxies with known analytic forms to
confirm that \Glass is able to correclty recover the mass profile.

\subsection{Mock galaxy data}

The four mock galaxies we consider have been designed to test our recovery
procedure. They are two component models where the dark matter and stellar
profiles are allowed to be steep and shallow.  The enclosed mass in all cases
is fixed at $M(a_\star)= 1.8\e{10}$ at the scale radius $a_\star=2$ kpc. These
values were chosen to closely resemble B1115. We therefore place the galaxy at
a redshift of $z_L = 0.31$ for lensing.  Throughout, we assume a cosmology
where $H_0^{-1}=13.7$ Gyr, $\Omega_M=0.28$, and $\Omega_\Lambda=0.72$. The
critical density is $\kappa_\mathrm{crit}\sim 1.8\e{9}$\Msun/kpc$^2$.

The galaxies were
first generated as three dimensional particle distributions with the
tool DEHNEN-TOOL. Each component follows the profile
\begin{equation}
\rho(r) = \frac{M}{4\pi a^3}(3-\gamma){(r/a)^{-\gamma}(1 + r/a)^{\gamma-4}}
\label{Dehnen profile}
\end{equation}
where $\gamma$ is the profile index and $a$ is the scale radius.  In the
case $\gamma=1$ this is the Hernquist profile.  The four combinations of
profile indices are shown in \tabref{mock galaxy params}.  Note that two of
the galaxies have $\gamma_\mathrm{DM}=1$. While we know from simulations that a
more accurate approximation would be an NFW or Einasto profile we are not
probing the outer regions of the halo where the difference would be most
extreme.  The Hernquist profile is easier to generate as it has finite mass.

\begin{table}
\begin{tabular}{cllllllll}
Galaxy & $\gamma_\star$ & $M_\star$ & $\gamma_\mathrm{DM}$ & $M_\mathrm{DM}$ & $\Rmap$ & Notes\\
\hline
AA & 1 & 4 & 0.05 & $11^{2.95}$ & 50 kpc & \\
AC & 1 & 4 & 1 & $11^2$ & 50 kpc & \\
BB & 1.5 & $2^{1.5}$ & 0.16 & $11^{2.84}$ & 50 kpc & \\
BC & 1.5 & $2^{1.5}$ & 1 & $11^2$ & 10 kpc & 
\end{tabular}
\caption{Profile parameters for the four mock galaxies. The resulting profiles only roughly follow
\eqnref{Dehnen profile} because the galaxies are triaxial. Masses are in units of $1.8\e{10}\Msun$. The scale lengths for
all lenses are $(a_\star,a_\mathrm{DM})=(2,20)$ kpc and by definition
$M_\star(a_\star) = M_\mathrm{DM}(a_\star) = 1$. $\Rmap$ is the 2d projected radius used to generate the lens configurations.}
\label{mock galaxy params}
\end{table}


\begin{figure}
\plotthree{MockGalProfile-a.pdf} {MockGalProfile-b.pdf} {MockGalProfile-c.pdf}
%\plotone{MockGalProfile-a.pdf} \plotone{MockGalProfile-b.pdf} \plotone{MockGalProfile-c.pdf}
\caption{
\textbf{(Left)} 
Three-dimensional density of the galaxies showing the stellar, dark matter,
and total densities with dotted, dashed, and solid lines, respectively. The
grey lines show the analytic forms from \eqnref{Dehnen profile}. 
\textbf{(Middle)} 
The radially averaged two-dimensional projected density.
The critical density for lensing at $z_L=0.31$ is $\kappa_\mathrm{crit}\sim 1.8\e{9}$\Msun/kpc$^2$.
\textbf{(Right)}
The enclosed projected mass.
}
\label{mock galaxies}
\end{figure}

In \figref{mock galaxies} we show the 3D radial density, 
the 2D projected density, and the 2D enclosed mass for each
galaxy. Solid lines represent the total mass, dashed lines the dark 
matter, and dotted lines the stellar mass. The projected plot has
been normalized to the critical density.

\subsection{Lens configurations} %--------------------------------------------------------------

For each of the four galaxies, we used the raytracing feature of \Glass\
described in \secref{Raytracing} to construct 12 lensing scenarios:

\begin{enumerate}
\item One double and one extended double.
\item One quad and one extended quad.
\item Two 2-source quads with varying redshift contrast.
\end{enumerate}

The extended configurations use extra sources to generate extra images that
simulate an arc. \figref{config ex.} shows the configuration for the extended
quad.

Each of these configurations were modeled with and without time delays and with
and without a central image. We assumed for all our tests that the lensing mass
was radially symmetric. By construction, this is known to be true and this is
most often the case with real galaxies, unless there is an obvious observed
asymmetry. The central pixel was refined into a further nine pixels to capture
any steep rise in the profile. Two of the four mock galaxies have a steeply rising
inner profile.

\figref{reconstruction} shows a typical reconstruction of a
lens. The far left plot shows the ensemble average arrival time surface with
image marked as circles and the inferred source position as a diamond. The
center plot shows the radial density profile. The error bars cover the full
range of models generated by \Glass. The true density profile from the mock
data is also plotted for comparison. The vertical lines mark the radial
position of the images. The final plot on the right is of the enclosed mass. As
expected the error bars are smallest in the region of the images where the most
information about the lens is present. The dip in the profile at the end of the
profile is due to the cut off in mass in the lensing map. This is of little
importance, though, as there is no lensing information there.

\begin{figure}
%\plotone{AAZContrastR1R5_Tm.pdf}
\plotthree{BCQuadR1a_TmS-a.pdf}{BCQuadR1a_TmS-b.pdf}{BCQuadR1a_TmS-c.pdf}
\caption{A typical reconstructed lens. Here we present a single quad lens from 
the BC mock galaxy.
\textbf{(Left)}
The arrival time surface. 
\textbf{(Middle)}
The surface density. The magenta curve represents the dark matter component,
the yellow curve the stellar component, and the blue curve is the sum of the two.
The black curve comes from the original mass model used to create the lens.
The green vertical lines mark the radial positions of the images. The higher
resolution feature of \Glass\ has been used on the central pixel allowing the
steep profile to be captured.
\textbf{(Right)}
The cumulative mass.}
\label{reconstruction}
\end{figure}

The main results from modeling these different configuration are shown black in
\figref{main results}. Each subplot corresponds to a different galaxy and
the vertical axis shows the range of quality.  The quality of a
model $\M_i$ by comparing the recovered density profile $\rho_\M(r)$ against
the profile of the mock galaxy $\rho_G(r)$.  In particular, the plots show the
RMS quantity
%
\begin{equation}
  \chi_i = 100 \times \sqrt{\frac{\int_r (\rho_{\M_i}(r) - \rho_G(r))^2}{\int_r \rho_G(r)^2}}
  %\chi_i = \frac{\sum_r (\rho(r)_{\M_i} - \rho(r)_G)^2}{\sum_r \rho(r)_G^2}
\end{equation}
%

\begin{figure}
\plottwo{AAchi2_profile.pdf}{BBchi2_profile.pdf}

\plottwo{ACchi2_profile.pdf}{BCchi2_profile.pdf}
\caption{The main results showing the quality of model recovery. Each panel corresponds to 
the named mock galaxy, whose parameters are listed in \tabref{mock galaxy params}. Within
each panel are six groups of results for each of six lens morphologies. Each morphology
considered the presence of time delays and a central image. The black markers are for tests
that did not include the stellar mass as a lower bound constraint, while the red markers
indicate where the stellar mass has been given.}
\label{main results}
\end{figure}

\begin{figure}
%\plotone{AAarrival_surfaces}
%\plotone{BBarrival_surfaces}
\plotone{BCarrival_surfaces}
\caption{The lens configurations for the six test cases using the BC mock galaxy. Here,
the central image is shown, although not all tests include it. The central image belongs
to only one set of images to avoid overconstraining the models. Similar colors group
images that share a common source. In the case of the extended image systems the source
is at the same redshift, while the redshift is varied between the systems in the 
ZConstrast cases. Grey circles are a visual aid to help determine radial separation
between images. The axes are measure in arcseconds.}
\label{reconstruction}
\end{figure}

The morphological complexity increases from left to right within each plot. As
a result, there is a general trend for the reconstruction quality to increase
(and for $\chi$ to decrease). By adding more measurable information to each
configuration the quality can also be affected. When both time delays and a
central image are present the quality is highest. A double is known to provide
very little constraint on the mass distribution. This is particular evident in
galaxies AC and BC where the mass profile is steepest and the reconstruction of
the double is poorest. However, the addition of an arc is sufficient to correct
this.

\subsection{Stellar mass}
\label{stellar mass}

The previous tests have been conducted to show that \Glass\ performs as expected.
Here, we present a novel constraint that can significantly improve the models. 
The stellar mass distribution is a lower bound on the total mass and in the
central region of a galaxy is also the dominant component. It can therefore be
used to constrain the innermost profile. We took the stellar mass directly from
the generated galaxies and projected the particles onto the pixels. \Glass\
also offers an option to interpolate any map of stellar mass (e.g., from an
observation) onto the pixels. The linear constraint is added to \Glass\ by
writing $\kappa_n = \kappa_{dm,n} + \kappa_{s,n}$ as the sum of the
dark matter and stellar mass components in the potential (\eqnref{discrete
potential}). Since each $\kappa_{s,n}$ is just a constant we do not add new,
separate equations for each pixel. 

With the stellar mass lower bound, the improvement of the reconstruction
quality shown in \figref{Test reconstructions} is quite dramatic for the
steepest mock galaxies (AC and BC). This is because these models are dominated
by stars in the inner region. The other two galaxies are unaffected because the
models were already well modeled by the dark matter alone.

\subsection{Velocity anisotropy}

In addition to the lens model aspects, \Glass\ can also run post processing
routines to analyze each model as it is produced. We have developed a number of
routines for fitting analytic models to the lensing mass profiles. One in
particular deprojects the 2d profile assuming a Hernquist profile for the
light. By fixing the velocity anisotropy $\beta$ we can determine the velocity
dispersion of the model. Of course, the true velocity dispersion is known for
the mock galaxies and we can compare our method to those measurements.  In
\figref{sigma-beta} we plot the recovered velocity dispersion for $\beta=0$ and $\beta=1$
over a range of aperture sizes for the BC Quad with time delays and stellar
mass. The stellar half mass radius is plotted in yellow and the Einstein radius
in black. For this configuration the two radii are well separated. This plot
demonstrates that the velocity anisotropy can be inferred by combining lens
modeling and independent velocity dispersion measurements. The lens breaks
the anisotropy degeneracy at a given aperture.

\begin{figure}
\plotone{BCQuadR1a_TmS-sb.pdf}
\caption{Estimated velocity dispersion as a function of the aperture size. The
single quad lens model for the BC mock galaxy was deprojected and the mass
profile fit assuming a Hernquist profile for the light. The top curve (blue)
assumes a velocity anisotropy $\beta=1$, while the lower curve (green)
assumes $\beta=0$. Given an independent estimate of the velocity dispersion,
lensing can be used to place constraints on the velocity anisotropy.}
\label{sigma-beta}
\end{figure}

%\subsection{Velocity dispersions}
%
%\begin{itemize}
%\item Can they predict radial profiles?
%\item Can we constrain lensing models from predicted velocity dispersions?
%\item Use a constructed spherical halo G2 following a $\rho \propto r^{-\gamma}$ profile with stellar halo same as G1.
%\item Vary $z$ of the halo, possibly orientation.
%\item Should be able to recover $\gamma$ very well.
%\item Use G1. Show problems with recovering $\gamma$.
%\end{itemize}

%\tabref{Test configurations} summarizes how each of the configurations is composed.  
%\figref{Test reconstructions} shows the reconstructed lens.

%\subsection{Recovery} %--------------------------------------------------------------

%\begin{table}
%\begin{tabular}{lccccc}
%Source name & $z$ & $\beta$ & Image & $\vec\theta$ & $\Delta t$ (days) \\
%\hline
%AASingleSourceR1a & 1.72 & -0.59, 0.10 & A & -2.64, 5.07 & \\
% & & & B & -3.15, -4.62 & 37.09\\
% & & & C & -5.72, -0.67 & 33.58\\
% & & & D & 4.26, -0.20 & 259.75\\
% & & & E & 0.22, -0.02 & 104.48\\
%AASingleSourceR1b & 1.72 & 0.11, -0.46 & A & 0.49, -5.85 & \\
% & & & B & 0.87, 4.91 & 228.69\\
% & & & C & 4.80, 1.57 & 31.83\\
% & & & D & -4.58, 1.40 & 56.15\\
% & & & E & -0.09, 0.11 & 180.09\\
%AAExtendedSourceR1a & 1.72 & -0.23, -0.13 & A & -1.08, -5.48 & \\
% & & & B & -1.26, 5.19 & 66.27\\
% & & & C & -5.31, 0.49 & 74.72\\
% & & & D & 4.72, 0.33 & 99.26\\
% & & & E & 0.04, 0.02 & 174.11\\
%AAExtendedSourceR1b & 1.72 & -0.37, 0.04 & A & -1.73, 5.27 & \\
% & & & B & -1.93, -5.13 & 14.76\\
% & & & C & -5.53, -0.27 & 75.45\\
% & & & D & 4.56, -0.05 & 160.47\\
% & & & E & 0.09, -0.01 & 147.24\\
%AAExtendedSourceR1c & 1.72 & -0.62, -0.13 & A & -2.84, -5.01 & \\
% & & & B & -3.47, 4.36 & 60.96\\
% & & & C & -5.66, 0.92 & 17.14\\
% & & & D & 4.13, 0.23 & 270.45\\
% & & & E & 0.40, 0.06 & 83.81\\
%AASingleSourceR2 & 1.50 & 0.11, -0.46 & A & 0.49, -5.66 & \\
% & & & B & 0.98, 4.67 & 227.84\\
% & & & C & 4.48, 1.70 & 27.39\\
% & & & D & -4.34, 1.34 & 54.57\\
% & & & E & -0.12, 0.21 & 144.98\\
%AASingleSourceR3 & 1.00 & 0.11, -0.46 & A & 0.48, -5.05 & \\
% & & & B & 0.92, 3.98 & 226.30\\
% & & & C & 3.70, 1.67 & 19.74\\
% & & & D & -3.55, 1.20 & 53.34\\
% & & & E & -0.12, 0.15 & 96.33\\
%AASingleSourceR4 & 0.80 & 0.11, -0.46 & A & 0.39, -4.50 & \\
% & & & B & 0.93, 3.38 & 223.14\\
% & & & C & 2.98, 1.69 & 9.75\\
% & & & D & -2.82, 1.09 & 48.86\\
% & & & E & -0.15, 0.18 & 52.90\\
%AASingleSourceR5 & 0.72 & 0.11, -0.46 & A & 0.44, -4.22 & \\
% & & & B & 1.02, 2.94 & 219.75\\
% & & & C & 2.51, 1.69 & 4.63\\
% & & & D & -2.36, 1.03 & 44.24\\
% & & & E & -0.16, 0.20 & 32.56
%\end{tabular}
%\end{table}


%\subsection{With/without time delays and central image} %--------------------------------------------------------------

%We also considered the effect of having time delays or a central image. The
%central image is usually highly demagnified and obscured by the lensing galaxy,
%but in clusters the central image is observable. In this case we can imagine
%the galaxies to be clusters since the problem is scale-free. A recontruction of
%AASingleQuadR1a\_Tm is shown in \figref{AASingleQuadR1a_Tm}. The suffix denotes
%which combination of time delays (T) and central image (M) was used. An
%uppercase letter indicates the feature is turned on, and a lowercase letter
%that it is turned off.

%\begin{figure}
%\plotone{AASingleQuadR1a_Tm.png}
%\caption{The reconstruction of AASingleQuadR1a\_Tm.}
%\end{figure}

%\begin{table}
%\begin{tabular}{lccccccc}
%Source name & $z$ & $\beta$ & $A$ (min) & $B$ (min) & $C$ (saddle) & $D$ (saddle) & $E$ (max)\\
%\hline
%SingleQuad1 & 2.50 & -0.37,0.08 & -2.40,10.40 & -2.55,-10.26 & -10.93,-0.39 & 10.37,-0.23 & 0.03,-0.01\\
%SingleQuad2 & 0.40 & 0.08,-0.51 & 0.32,-7.23 & 0.70,6.40 & 6.30,1.74 & -6.22,1.59 & -0.03,0.08\\
%SingleQuad3 & 0.60 & 0.08,-0.51 & 0.27,-8.86 & 0.74,8.08 & 8.13,1.93 & -8.05,1.72 & -0.03,0.06\\
%SingleQuad4 & 0.80 & 0.08,-0.51 & 0.36,-9.58 & 0.70,8.82 & 8.92,2.05 & -8.87,1.76 & -0.02,0.05\\
%SingleQuad5 & 1.00 & 0.08,-0.51 & 0.38,-10.00 & 0.68,9.24 & 9.40,1.99 & -9.34,1.79 & -0.02,0.05\\
%ExtendedSource0 & 0.40 & 0.20,0.57 & 0.79,7.25 & 1.51,-6.25 & 6.29,-2.07 & -6.12,-1.55 & -0.06,-0.08\\
%ExtendedSource1 & 0.40 & -0.15,0.17 & -0.64,6.96 & -0.84,-6.66 & -6.67,-0.56 & 6.39,-0.52 & 0.02,-0.03\\
%ExtendedSource2 & 0.40 & 0.06,-0.33 & 0.24,-7.09 & 0.44,6.56 & 6.47,1.18 & -6.43,0.98 & -0.03,0.05
%\end{tabular}
%\caption{Source data.}
%\label{source data}
%\end{table}
%
%\begin{table}
%\begin{tabular}{lrrrr}
%Source name & $\Delta t_{AB}$ & $\Delta t_{BC}$ & $\Delta t_{CD}$ & $\Delta t_{DE}$\\
%\hline
%SingleQuad1 & 38.97 & 171.07 & 207.10 & 1314.05\\
%SingleQuad2 & 335.01 & 97.91 & 50.78 & 623.95\\
%SingleQuad3 & 312.86 & 122.63 & 47.83 & 951.56\\
%SingleQuad4 & 303.39 & 128.81 & 46.63 & 1102.50\\
%SingleQuad5 & 298.27 & 131.32 & 45.70 & 1188.64\\
%ExtendedSource0 & 360.17 & 62.16 & 115.30 & 588.15\\
%ExtendedSource1 & 109.28 & 169.77 & 91.20 & 620.78\\
%ExtendedSource2 & 217.40 & 148.57 & 36.52 & 642.27
%\end{tabular}
%\caption{Time delays for each image system.}
%\label{source time delays}
%\end{table}
%
%\begin{table}
%\begin{tabular}{ll}
%\label{test systems}
%Test name & Source name\\
%\hline
%SingleQuad & SingleQuad1\\
%SingleQuad2 & SingleQuad2\\
%ExtendedSource & ExtendedSource0 + ExtendedSource1 + ExtendedSource2\\
%ZContrastDoubleQuad0 & SingleQuad1 + SingleQuad2\\
%ZContrastDoubleQuad1 & SingleQuad1 + SingleQuad3\\
%ZContrastDoubleQuad2 & SingleQuad1 + SingleQuad4\\
%ZContrastDoubleQuad3 & SingleQuad1 + SingleQuad5
%\end{tabular}
%\caption{Test configurations}
%\label{Test configurations}
%\end{table}

\section{Results} %--------------------------------------------------------------

%\subsection{Model quality}

%\subsection{Shape fitting}

%\subsection{Using principle component analysis to find new degeneracies}

%\begin{figure}
%\label{chi2}
%\plottwo{AAchi2_profile.pdf}{AAchi2_profile.pdf} \\
%\plottwo{BBchi2_profile.pdf}{BCchi2_profile.pdf}
%\caption{
%Goodness of fit plots for the reconstruction of each of the four galaxies given
%the different lensing test cases. The $\chi^2$ is computed over the radial $\kappa$
%profile with data points at the median value and the error bars showing the 68\% confidence
%interval.
%}
%\end{figure}

%\begin{figure}
%\plotone{SingleQuad_TmC.png}
%\caption{SingleQuad}
%\label{SingleQuad}
%\end{figure}
%
%\begin{figure}
%\plotone{SingleQuad2_TmC.png}
%\caption{SingleQuad2}
%\label{SingleQuad2}
%\end{figure}
%
%\figref{SingleQuad}
%\figref{SingleQuad2}
%
%\begin{figure}
%\plotone{ExtendedSource_TmC.png}
%\caption{ExtendedSource}.
%\label{ExtendedSource}
%\end{figure}
%
%\figref{ExtendedSource}
%
%\begin{figure}
%\plotone{ZContrastDoubleQuad0_TmC.png}
%\caption{ZContrastDoubleQuad0}
%\label{ZContrastDoubleQuad0}
%\end{figure}
%
%\begin{figure}
%\plotone{ZContrastDoubleQuad1_TmC.png}
%\caption{ZContrastDoubleQuad1}
%\label{ZContrastDoubleQuad1}
%\end{figure}
%
%\begin{figure}
%\plotone{ZContrastDoubleQuad2_TmC.png}
%\caption{ZContrastDoubleQuad2}
%\label{ZContrastDoubleQuad2}
%\end{figure}
%
%\begin{figure}
%\plotone{ZContrastDoubleQuad3_TmC.png}
%\caption{ZContrastDoubleQuad3}
%\label{ZContrastDoubleQuad3}
%\end{figure}
%
%\figref{ZContrastDoubleQuad0}
%\figref{ZContrastDoubleQuad1}
%\figref{ZContrastDoubleQuad2}
%\figref{ZContrastDoubleQuad3}

%%%%%%%%%%%%%%%%%%%%%%%%%%%%%%%%%%%%%%%%%%%%%%%%%%%%%%%%%%%%%%%%%%%%%%%%%%%%%%%%
%%%%%%%%%%%%%%%%%%%%%%%%%%%%%%%%%%%%%%%%%%%%%%%%%%%%%%%%%%%%%%%%%%%%%%%%%%%%%%%%
%%%%%%%%%%%%%%%%%%%%%%%%%%%%%%%%%%%%%%%%%%%%%%%%%%%%%%%%%%%%%%%%%%%%%%%%%%%%%%%%
%%%%%%%%%%%%%%%%%%%%%%%%%%%%%%%%%%%%%%%%%%%%%%%%%%%%%%%%%%%%%%%%%%%%%%%%%%%%%%%%

%\subsection{Aperture}

%\begin{figure}
%\plotone{sigma_beta.pdf}
%\caption{}
%\end{figure}

\section{Discussion} %--------------------------------------------------------------

%%%%%%%%%%%%%%%%%%%%%%%%%%%%%%%%%%%%%%%%%%%%%%%%%%%%%%%%%%%%%%%%%%%%%%%%%%%%%%%%
%%%%%%%%%%%%%%%%%%%%%%%%%%%%%%%%%%%%%%%%%%%%%%%%%%%%%%%%%%%%%%%%%%%%%%%%%%%%%%%%
%%%%%%%%%%%%%%%%%%%%%%%%%%%%%%%%%%%%%%%%%%%%%%%%%%%%%%%%%%%%%%%%%%%%%%%%%%%%%%%%
%%%%%%%%%%%%%%%%%%%%%%%%%%%%%%%%%%%%%%%%%%%%%%%%%%%%%%%%%%%%%%%%%%%%%%%%%%%%%%%%


%\begin{figure}
%\plotone{SingleQuad_Tmc_eig.png}
%\plotone{SingleQuad2_Tmc_eig.png}
%\plotone{ExtendedSource_Tmc_eig.png}
%\plotone{ZContrastDoubleQuad0_Tmc_eig.png}
%\plotone{ZContrastDoubleQuad1_Tmc_eig.png}
%\plotone{ZContrastDoubleQuad2_Tmc_eig.png}
%\plotone{ZContrastDoubleQuad3_Tmc_eig.png}
%\end{figure}

\appendix


%%%%%%%%%%%%%%%%%%%%%%%%%%%%%%%%%%%%%%%%%%%%%%%%%%%%%%%%%%%%%%%%%%%%%%%%%%%%%%%%
%%%%%%%%%%%%%%%%%%%%%%%%%%%%%%%%%%%%%%%%%%%%%%%%%%%%%%%%%%%%%%%%%%%%%%%%%%%%%%%%
%%%%%%%%%%%%%%%%%%%%%%%%%%%%%%%%%%%%%%%%%%%%%%%%%%%%%%%%%%%%%%%%%%%%%%%%%%%%%%%%
%%%%%%%%%%%%%%%%%%%%%%%%%%%%%%%%%%%%%%%%%%%%%%%%%%%%%%%%%%%%%%%%%%%%%%%%%%%%%%%%

%\section{Choice of modeling parameters}
%
%
%\begin{itemize}
%\item \figref{raytracing convergence tests}
%\item Why do we choose {\texttt maprad}=20 arcsec when generating test cases?
%We need to enclose most of the mass for the time delays to be stable.  
%\item Why do we choose {\texttt pixrad}=45 when generating the test cases? We
%need sufficient resolution for the image positions to be stable.  
%\end{itemize}


%%%%%%%%%%%%%%%%%%%%%%%%%%%%%%%%%%%%%%%%%%%%%%%%%%%%%%%%%%%%%%%%%%%%%%%%%%%%%%%%
%%%%%%%%%%%%%%%%%%%%%%%%%%%%%%%%%%%%%%%%%%%%%%%%%%%%%%%%%%%%%%%%%%%%%%%%%%%%%%%%
%%%%%%%%%%%%%%%%%%%%%%%%%%%%%%%%%%%%%%%%%%%%%%%%%%%%%%%%%%%%%%%%%%%%%%%%%%%%%%%%
%%%%%%%%%%%%%%%%%%%%%%%%%%%%%%%%%%%%%%%%%%%%%%%%%%%%%%%%%%%%%%%%%%%%%%%%%%%%%%%%

%\section{GLASS - An Extensible Gravitational Lens Modeling Framework}
%\label{GLASS Description}
%\begin{itemize} 
%\item Framework that incorporates lens modeling, analysis, and plotting.
%\item Not limited to just pixels based modeling. Can be used with functional forms like Bessel functions. Not yet implemented.
%\item Allows for post processing and criteria matching on a per model basis.
%\item Input files are python programs but from within the framework that aspect may be ignored.
%
%\item Lens reconstruction. Based on PixeLens. Express data constraints and
%priors as linear equations. Sample the resulting simplex.
%\item Standard data constraints: image positions, time delays, hubble constant, parity, annular density, external shear.
%\item Priors: range of profile steepness, local gradient, smoothness, shared H0, symmetry for doubles.
%\item Local gradient and smoothness are now convergent. Based on physical size rather than number of neighbors.
%
%\item Raytracing. Use a fine grid to sample the 2d function mapping position on
%the image plane to distance from the source on the source plane. From those
%points that are near the source, use a 2d root finder to locate the exact
%positions. Merge nearly identical solutions.

%\end{itemize}

\section{Raytracing convergence test}
Raytracing is sensitive to the map area and resolution set by $\Rmap$ and
$\Rpix$.  In the left panel of \figref{raytracing convergence tests} we compare
the change in time delays as we increase $\Rmap$.  The image system is a quad
with the central maximum image. When the projected mass begins to fall off the
time delays stabilize. In the right panel we see the effect of changing the
resolution of the map.  Increasing the resolution allows for more accurate
placement of the images and after about $\Rpix=40$ the image positions change
by less than 0.04 arcsec. The one image that continues to change is near to a
caustic.  

\begin{figure}
\plottwo{tdconv_pr45.pdf}{imgpos_conv_mr20.pdf}
\caption{(left) Test for predicted time delay convergence as $\Rmap$ changes.
$\Rpix=45$. After about $\Rmap=20$ most of the asymmetric mass is in the map.
(right) Test for predicted image position $\theta$ convergence as $\Rpix$
changes. $\Rmap=20$ arcsec.}
\label{raytracing convergence tests}
\end{figure}

\section{Derivation of pixelated density coefficients}
\label{Q derivation}
When the lens plane is pixelized we need a discrete form of the integral
%
\[\int \kappa(\vec\theta') \ln |\vec\theta-\vec\theta'| d^2\vec\theta' \]
%
In particular we want
%
\[\sum_n \kappa_n Q_n(\vec\theta)\]
%
where $Q_n$ is the logarithm evaluated over the $n$th pixel at position $\vec\theta_n = (x_n, y_n)$. Let the pixel side length be $a$.
Instead of working with a position vector $\vec\theta$ we work in cartesian coordinates such that
%
$|\vec\theta| = r = \sqrt{x^2 + y^2}$. The integral now becomes
%
\[Q_n(x,y) = \frac12 \int_{y_-}^{y_+}\int_{x_-}^{x_+} \ln (x'^2+y'^2) dx' dy'\]
%
where $x_\pm = x + x_n \pm (a/2)$ and similarly for $y_\pm$.
Using the identity
%
\[\int \ln(x^2+y^2) dx = x \ln(x^2+y^2) - 2x + 2y\arctan(x/a) \]
%
we can express $Q_n$ as the sum of four parts
%
\[Q_n(x,y) = \frac12 \left[ \tilde Q_n(x_+,y_+)
        + \tilde Q_n(x_-,y_-)
        - \tilde Q_n(x_-,y_+)
        - \tilde Q_n(x_+,y_-) \right]\]
%
where
%
\[\tilde Q_n(x,y) = xy(\ln r^2 - 3) + x^2\arctan(y/x) + y^2\arctan(x/y)\]

\bibliographystyle{mn2e}
\bibliography{ms}

\end{document}

