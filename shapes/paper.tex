\documentclass[useAMS,usenatbib]{mn2e}
\usepackage{myaasmacros}
\usepackage{graphicx}
\usepackage{ulem}
\usepackage{amsmath}
\usepackage{multirow}
\usepackage{color}
\usepackage{amssymb}

\def\Glass{{\sc Glass}}
\def\PixeLens{{\sc PixeLens}}

\title[Light versus dark in strong-lens galaxies]{Light versus dark in strong-lens galaxies: Dark matter haloes that are rounder than their stars}

\author[Bruderer et al.]
{\parbox{\textwidth}{Claudio Bruderer,$^{1}$\thanks{E-mail: \texttt{claudio.bruderer@phys.ethz.ch}}
Justin I. Read,$^{1,2}$
Jonathan P. Coles,$^{3,4}$
Dominik Leier,$^{5}$
Emilio E. Falco,$^{6}$
Ignacio Ferreras$^{7}$ and
Prasenjit Saha$^{4,8}$}\vspace{0.4cm}\\
\parbox{\textwidth}{$^{1}$Institute for Astronomy, Department of Physics, ETH Zurich, Wolfgang-Pauli-Strasse 27, 8093 Z\"urich, Switzerland\\
$^{2}$Department of Physics, University of Surrey, Guildford, GU2 7XH, UK\\
$^{3}$Exascale Research Computing Lab, Campus Teratec, 2 Rue de la Piquetterie, 91680 Bruyeres-le-Chatel, France\\
$^{4}$Physik-Institut, University of Zurich, Winterthurerstrasse 190, 8057 Z\"urich, Switzerland\\
$^{5}$Dipartimento di Fisica e Astronomia, Universit\`{a} di Bologna, viale Berti Pichat 6/2, 40127, Bologna, Italy\\
$^{6}$Harvard-Smithsonian Center for Astrophysics, 60 Garden St., Cambridge, MA 02138, USA\\
$^{7}$Mullard Space Science Laboratory, University College London, Holmbury St Mary, Dorking, Surrey RH5 6NT, UK\\
$^{8}$Institute for Computational Science, University of Zurich, Winterthurerstrasse 190, 8057 Z\"urich, Switzerland}}


\begin{document}

\maketitle

\begin{abstract}
We measure the shape and alignment of the stellar and dark matter mass distribution in 11 strong-lens galaxies, finding over the range of several multiples of the half-light radii that the dark matter haloes are rounder than the stellar mass distributions. Averaging over larger radii, the ellipticity, as defined using the ratio of the semi-minor and semi-major axes, increases both in their dark matter and stellar components. While dark matter haloes are never more elliptical than $e_{dm} = 0.3$, their stars can extend to $e_* > 0.3$. Three systems, in particular, have a high stellar ellipticity ($e_* > 0.4$) and correspondingly a high alignment between the luminous and the dark matter distribution. One of these -- {\it B1608+656} -- is a known merging pair; we suggest that the other two ({\it B0712+472} and {\it B2016+112}) may also be recent post-merger systems. Galaxies with high dark matter ellipticity and weak external shear show strong alignment between light and dark; those with strong shear ($\gamma \gtrsim 0.1$) can be highly misaligned. This is reassuring since isolated misaligned galaxies are expected to be unstable.

Our results provide a new constraint on galaxy formation models that must explain the origin of both very round dark matter haloes and misaligned strong-lens systems.

\end{abstract}

\begin{keywords}
Gravitational lensing: strong -- galaxies: structure -- galaxies: haloes -- galaxies: formation -- galaxies: elliptical and lenticular, cD.
\end{keywords}


\section{Introduction}\label{sec:introduction}

The ellipticity and shape of the stellar component relative to their host dark matter halo encodes information both about our current cosmological model $\Lambda$CDM and galaxy formation \citep[e.g.][]{1994ApJ...431..617D,2001ApJ...551..294I,2004ApJ...611L..73K,2007MNRAS.378...55M,2007arXiv0707.0737D,2012MNRAS.424L..16L,2014JPhG...41f3101R}. `Dark-matter-only' (DMO) simulations in $\Lambda$CDM predict dark matter haloes that are triaxial \citep{1991ApJ...378..496D,1992ApJ...399..405W,1996ApJ...462..563N,2002ApJ...574..538J}, with mean `shape parameter' $\langle q \rangle = (b+c)/2a \sim 0.8$ (where $a > b > c$ are the long, intermediate and short axes of the figure; \citealt{2007MNRAS.378...55M}). This corresponds to a typically {\it prolate} halo. However, including `baryons' (stars and gas) in the models produces haloes that are significantly rounder and -- at least for disc galaxies -- well-aligned with the light distribution \citep{1991ApJ...377..365K,1994ApJ...431..617D,2007arXiv0707.0737D}. Halo shapes and alignments also constrain alternative gravity models \citep{2001MNRAS.327..552M,2004ApJ...610L..97H,2005MNRAS.361..971R,2012PhRvD..86h3507F,2013MNRAS.434.2971D}. If stars dominate the mass of the galaxy, we expect the light and mass distribution to be highly correlated; if dark matter is present, however, such correlations can, at least in principle, be broken.

Strong lensing provides a unique probe of the alignment and shape of the total mass distribution in galaxies \citep[e.g.][]{1986ApJ...310..568B,1992grle.book.....S,1998ApJ...509..561K,2000ApJ...543..131K,2006ApJ...649..599K,2007AJ....134..668A,2008MNRAS.383..857F,2010ApJ...724..511A,2012MNRAS.424..104L}. For `red and dead' ellipticals that are largely devoid of gas, their baryonic content can be mapped through stellar population synthesis modelling of their light distribution alone \citep[e.g.][]{2005ApJ...623L...5F,2006ApJ...640..662T,2008MNRAS.383..857F}. Furthermore, such systems are dense enough to produce strong lensing effects, opening up the possibility of directly comparing the light and mass in these galaxies \citep{1998ApJ...509..561K,2008MNRAS.383..857F,2009ApJ...690..670T,2012A&A...538A..99S}. Previous work in the literature has found that the light and mass are well-aligned (though a mis-match of up to 10$^\circ$ is not uncommon; e.g. \citealt{2012A&A...538A..99S}). However, results on the ellipticity of light and mass agree less well, with \citet{2012A&A...538A..99S} finding a strong correlation and \citet{1998ApJ...509..561K} and \citet{2008MNRAS.383..857F} finding none. It is difficult, however, to compare the results between these different studies because they use different lens modelling techniques; different definitions of ellipticity; and different radii over which the shapes and alignments are probed. Furthermore, none to date have applied their methodology to mock data to determine the robustness of the results.

Weak lensing can also be used to probe the shape and alignment between the luminous and dark matter distributions, but only for `stacked' galaxies \citep{2000astro.ph..6281B,2000ApJ...538L.113N}. \citet{2004ApJ...606...67H} applied this idea to data from the Red-Sequence Cluster Survey (RCS) to measure the first weak lensing signal of halo flattening. They found that dark matter haloes appear to be rounder than the stellar mass distribution, with some weak evidence for alignment. Both measurements are challenging however, and more recent data appear to be at odds with this early work, at least for galaxies in general. They however are in agreement concerning `red' galaxies, which are predominantly early-type galaxies, also favouring dark matter haloes that are more elliptical than their stellar mass components \citep{2006MNRAS.370.1008M,2007ApJ...669...21P,2012A&A...545A..71V}.

Recently, \cite{2014arXiv1401.7990C} introduced a new non-parametric lens tool, \Glass. Applying this to a large suite of mock data, we showed that mass and light can only reliably be disentangled in lens systems if: i) there are at least four images; and ii) time delay data are available and/or the stellar mass contributes significantly to the potential. In this paper, we collate data of the above quality, compiling a sample of 11 strong lens galaxies. We apply \Glass\ to these lenses to non-parametrically measure the shape and alignment of the stars and {\it dark matter} in these lens galaxies, for the first time. This differs from previous works that have all compared the light distribution with the total mass, rather than the dark matter. Since the stellar component often dominates the central potential, the total mass naturally correlates with the light, potentially masking theoretically interesting results about the dark matter distribution. Our comparison between the stellar and dark matter components is made possible by the fact that \Glass\ uses the stellar mass distribution as a prior on the mass map, ensuring that the dark matter map is always positive.

This paper is organised as follows. In \S\ref{sec:glass}, we briefly review the \Glass\ code and define our method to assess shape and alignment of lens galaxies. In \S\ref{sec:data}, we present our data compilation with references. In \S\ref{sec:results}, we present our results. Finally, in \S\ref{sec:conclusions} we discuss the implications of these results and we present our main conclusions.


\section{\Glass: A non-parametric lens tool}\label{sec:glass}

\Glass\ is a new `non-parametric' gravitational lens modelling framework \citep{2014arXiv1401.7990C} that generates a distribution of mass models given a set of basic assumptions, or priors. Hence, \Glass\ does not yield the most probable mass model. Rather, it calculates an ensemble of all allowed models given the data and assumed priors. In this context, `non-parametric' really just means that many more parameters than data constraints are used such that the system of equations describing the lensing mass distribution is under-constrained. \Glass\ shares some aspects with an earlier code \PixeLens\ \citep{2004AJ....127.2604S,2008ApJ...679...17C}, but has a completely different code base. Its key improvements upon \PixeLens{} relevant for this paper include:

\begin{enumerate}
\item A new uniform sampling algorithm for high dimensional spaces \citep{2012MNRAS.425.3077L} is employed. This allows for large ensembles of $\gtrsim$10,000 models to be efficiently generated. 
\item With so many models in the final ensemble, we can afford to apply non-linear constraints (for example stellar kinematic data; or the removal of models with spurious extra images) to accept/reject models in a post-processing step.
\item The central region of the mass map can have a higher resolution to more efficiently capture steep models.
\item Stellar density can be used as an additional lower-bound constraint on the models. 
\end{enumerate} 

In this paper, the default priors used to model each lens are listed below. Table \ref{tab:lenspriors} identifies lenses for which we have modified these defaults. For each lens we assume:

\begin{enumerate} 
\item Observational priors (redshifts, image positions relative to lens position).
\item Image parity to be enforced.
\item A reconstructed stellar mass map that provides a lower mass bound to each pixel.
\item The local density gradient to be always pointing within 50$^{\circ}$ of the centre of the distribution, and the circularly-averaged density to never increase outwards.
\item The fixed, flat cosmology with matter density $\Omega_{m} = 0.28$, the dark energy density $\Omega_{\Lambda} = 0.72$, and the inverse Hubble constant $H_{0}^{-1} = 13.7$ Gyr.
\item A flat prior on the magnitude of the external shear $|\gamma|$ on $[0,1)$.
\item The radius of the lens to be set to 12 pixels.
\end{enumerate}

In cases where more information is available, the following priors are also used:

\begin{enumerate}
\item[(viii)] A flat prior on measured time delays within the $\pm 1\sigma$-interval.
\item[(ix)] Point symmetry $\kappa(x,y) = \kappa(-x,-y)$, where the surface mass profile can be assumed to be symmetric ((3+1)-quad configurations and the double \textit{Q0957+561}).

\end{enumerate}

\subsection{Measuring the shape and alignment of a lens}\label{sec:shapemethod}
We employ a shape definition related to the definition used by \cite{2014arXiv1401.7990C}. The ellipticity parameter $e$ is defined as
\begin{equation}\label{eq:shapeestimate}
    e \equiv 1 - \frac{\lambda_{2}}{\lambda_{1}},
\end{equation}
where $\lambda_{1}$ and $\lambda_{2}$ ($\lambda_{1} \geq \lambda_{2}$) are the eigenvalues of the inertia tensor $I$ of the mass distribution $M(\boldsymbol{\theta})$. It is defined as
\begin{equation}\label{eq:inertiatensor}
    I = \begin{pmatrix} \sum_\theta M(\boldsymbol{\theta})\theta^{2}_{y} & -\sum_\theta M(\boldsymbol{\theta})\theta_{x}\theta_{y} \\
                        -\sum_\theta M(\boldsymbol{\theta})\theta_{x}\theta_{y} & \sum_\theta M(\boldsymbol{\theta})\theta^{2}_{x} \end{pmatrix}.
\end{equation}
We use these estimators to quantify the shape of the stellar mass distribution and the dark matter distribution. We estimate the latter by reconstructing the total mass distribution with a lower mass prior from the stellar mass distribution and then subtracting the latter off. The shapes of the distributions are computed for all the pixels up to the outermost lensed image and for the pixels within different multiples of $R_e$, defined as the half-light radius measured with respect to the Petrosian flux. As terms in Eq.~\ref{eq:inertiatensor} are weighted with $r^2$, the outer pixels tend to dominate the estimate of the averaged shape within the considered radius. The shape estimates do not necessarily need to be the same within different radii -- e.g. effects such as isophotal twist \citep[e.g.][and references therein]{1978ComAp...8...27B} can cause differences in the shape estimates between the central and the outer parts of the system.

Throughout this paper, we generate in each case an ensemble of 10,000 mass models. Each of these models is analysed individually, yielding a distribution of constrained quantities.


\section{Data}\label{sec:data}
Stellar mass maps are taken from \cite{2011ApJ...740...97L}. In that work, the surface brightness distribution of the galaxy images in all available HST bands was first fitted using {\sc Galfit v2.03b} \citep{2002AJ....124..266P}, and with the resulting flux maps stellar mass maps using the synthesis of the stellar-population (SPS) models of \cite{2003MNRAS.344.1000B} were obtained. The population synthesis is marginalised over the star formation epoch and time scales and over the stellar metallicity. The initial mass function (IMF) was taken to be the log-normal form given by \cite{2003PASP..115..763C}; a more bottom-heavy IMF would change the mass normalisation \citep[cf.][]{2014ApJ...793...96S} but not the shape of the inferred stellar mass distribution \citep[unless the IMF presents significant intrinsic deviations locally, see e.g.][]{2015MNRAS.447.1033M}. We note that the effect of recent claims towards a non-universal IMF on stellar mass-to-light ratio (M/L) is still under debate. While for massive elliptical galaxies, dynamical studies seem to point at contributions at the level of the Salpeter IMF or slightly heavier \citep{2013MNRAS.432.1862C}, spectral line strength constraints can still accommodate a wider range of the stellar M/L depending on the functional form of the IMF \citep{2013MNRAS.429L..15F}. For at least one lens in our sample, \textit{Q2237+030}, the lensing estimates on the mass distribution are inconsistent with a Salpeter IMF \citep{2010MNRAS.409L..30F}. We additionally tested varying the IMF for one lens ({\it B1422+231}), finding that Salpeter IMF is too heavy and gives no solutions. We will perform a more systematic study of the IMF and varying stellar M/L in a forthcoming publication.

The original sample we consider, is presented in \cite{2011ApJ...740...97L} and consists of 21 lens systems. It is a subsample of the CfA-Arizona Space Telescope LEns Survey\footnote{http://www.cfa.harvard.edu/castles/} (CASTLES) sample. As shown in \cite{2014arXiv1401.7990C}, the recovery of the shape information of the lens is only good if there are at least four images. Therefore, we only consider the quad lenses and the twin double \textit{Q0957+561}. This yields a sample consisting of 11 lenses. We discuss each of these systems, next.

\textit{Q0047-2808} (hereafter \textit{0047}) is a luminous early-type galaxy \citep{1996MNRAS.278..139W}. It is part of a group of 9 members, all of which are spectroscopically confirmed \citep{2011ApJ...726...84W}.

\textit{MG0414+0534} (hereafter \textit{0414}) is a passively evolving early-type galaxy \citep{1999AJ....117.2034T}. A very close luminous satellite galaxy was found north-west of the lens \citep{1993AJ....105....1S}. It, however, is more likely a foreground object with a redshift of ~0.38 \citep{2011MNRAS.413L..86C}. This is sufficiently distant to still contribute to the lensing effect.

\textit{B0712+472} (hereafter \textit{0712}) is an early-type galaxy \citep{1998MNRAS.296..483J,1998AJ....115..377F}. There is a foreground group of about 10 member galaxies at $z\sim0.3$, which could affect the modelling \citep{2002AJ....123..627F}.

\textit{RXJ0911+0551} (hereafter \textit{0911}) is an almost spherical early-type galaxy \citep{1997A&A...317L..13B,2012A&A...538A..99S}, and has measured time delays \citep{2002ApJ...572L..11H}. It lies on the outskirts of a cluster of which X-ray emission can be detected yielding a temperature of 2.3 keV \citep{2001ApJ...555....1M}. There is also a satellite galaxy to the northwest \citep{2000ApJ...544L..35K}.

\textit{Q0957+561} (hereafter \textit{0957}) is a cD galaxy lying close to the centre of a cluster with a high spiral galaxy-fraction \citep[e.g.][]{1992MNRAS.254P..27G,1994A&A...291..411A,1998ApJ...504..661C}. Due to the large image separation, large physical scales are probed. As listed in Table 3 of \cite{2000ApJ...542...74K}, the position angles to the centre of the cluster of earlier lens reconstructions range between 51.8 deg and 67.8 deg. These results are consistent with the centre of the X-ray emission from the cluster \citep{1998ApJ...504..661C}. The lens has a measured time delay \citep[e.g.][]{2012A&A...540A.132S}. They however also find a three-day lag between the g- and r-bands, a disagreement at the 2$\sigma$-level. They argue that this effect can be accounted for by the presence of a substructure and chromatic dispersion. We find that the results do not change significantly for either estimate and choose therefore the g-band measurement.

\textit{PG1115+080} (hereafter \textit{1115}) is an early-type galaxy \citep{1980Natur.285..641W,2005ApJ...626...51Y}. It has measured time delays \citep[see e.g.][]{1997ApJ...475L..85S}. In this work, we use recent estimates of the time delays by \cite{2010MNRAS.406.2764T} that differ from previously found values by e.g. \cite{1997ApJ...489...21B}. The environment of the lens was analysed thoroughly \citep{2006ApJ...641..169M,2011ApJ...726...84W}. It is part of a small group of 13 members. Also, X-ray emission was detected from the corresponding group and it yields a temperature of 0.8 keV \citep{2004ApJ...610..686G}.

\textit{B1422+231} (hereafter \textit{1422}) is an early-type galaxy \citep{1992MNRAS.259P...1P,1994AJ....107...28Y}. Although time delays have been reported \citep{2001MNRAS.326.1403P}, it is possible that the measurements are aliases of the actual values \citep{2003AJ....126...29R} and hence are not included in our analysis. The lens is part of a group with 16 spectroscopically confirmed member galaxies \citep{2006ApJ...641..169M}. Using newer data, additional members could be identified \citep{2011ApJ...726...84W}. \cite{2004ApJ...610..686G} also detect X-ray emission from the corresponding group at a temperature of 1.0 keV.

\textit{B1608+656} (hereafter \textit{1608}) was first reported by \cite{1995ApJ...447L...5M}. It consists of two merging galaxies. The main galaxy is an early-type galaxy which is disrupted by a smaller, probably late-type galaxy \citep{2003ApJ...584..100S}. The system has measured time delays \citep{2002ApJ...581..823F}. The environment has been analysed and a group with 8 (9 if merging galaxies are counted individually) was identified \citep{2006ApJ...642...30F}. However, no significant X-ray emission was detected from the surrounding group \citep{2005ApJ...625..633D}. The data seem to indicate four other groups along the line of sight.

\textit{MG2016+112} (hereafter \textit{2016}) is a giant elliptical galaxy \citep{1984Sci...223...46L,1986AJ.....91..991S}. It is the farthest lens we consider in this sample. It lies in a cluster that consists of 69 probable galaxies \citep{2003MNRAS.344..337T}. The cluster shows a high density of galaxies close to the lens in a south-east direction. Two lensed images are very close in this fold lens, but could be resolved by \cite{2009MNRAS.394..174M}.

\textit{B2045+265} (hereafter \textit{2045}) is probably an elliptical galaxy \citep{2007MNRAS.378..109M}. It was initially classified as a late-type Sa galaxy \citep{1999AJ....117..658F}, however the velocity dispersion seems too high. As the source redshift is rather low, a large lens mass is required. \cite{2007MNRAS.378..109M} therefore conclude that it is more likely an elliptical galaxy. To the west of the lens, a group was found at a similar redshift as the lens \citep{1999AJ....117..658F}. There is furthermore evidence for a dwarf satellite galaxy \citep{2007MNRAS.378..109M}. As the measurements are however inconclusive, we choose to not include this satellite in our models.

\textit{Q2237+030} (hereafter \textit{2237}) is a barred spiral \citep{1988AJ.....95.1331Y}. With a redshift of just $z\sim0.04$ it is the closest lens system of the sample. Due to its low redshift, the probed physical scales are small, making the bulge component of the galaxy dominant. This makes the treatment of the stellar populations easier, as derivations of the stellar M/L are more difficult in the presence of star forming regions typical of disc galaxies, where there are no simple stellar populations usually assumed in SPS models.

Further information on the sample can be found in \citet{2011ApJ...740...97L} and \citet{2012A&A...538A..99S}. The most important quantities of each system are given in Table~\ref{tab:lensproperties}. Due to the different angular separations of the images and the different angular diameter distances, in each lens galaxy we necessarily probe slightly different scales.

\begin{table}
  \begin{center}
    \begin{tabular}{l r r r r l}
      Lens    & \multicolumn{1}{c}{$z_{L}$} & \multicolumn{1}{c}{$z_{S}$} & \multicolumn{1}{c}{$\Delta\theta$ [kpc]} & \multicolumn{1}{c}{$R_{L}$/$R_{e}$} & Environment \\ \hline
      0047 & 0.485 & 3.60 & 12.82 & $1.45\pm0.04$ & G(9) \\
      0414 & 0.960 & 2.64 & 16.01 & $1.85\pm0.05$ & ... \\
      0712 & 0.410 & 1.34 & 6.82  & $1.15\pm0.03$ & ... \\
      0911 & 0.769 & 2.8  & 23.16 & $3.09\pm0.05$ & C \\
      0957 & 0.356 & 1.41 & 29.98 & $3.51\pm0.04$ & C \\
      1115 & 0.310 & 1.72 & 10.76 & $2.86\pm0.06$ & G(13) \\
      1422 & 0.337 & 3.62 & 6.02  & $4.49\pm0.06$ & G(17) \\
      1608 & 0.630 & 1.39 & 13.92 & $1.82\pm0.01$ & G(8) \\
      2016 & 1.010 & 3.3  & 26.22 & $6.12\pm0.14$ & C(69) \\
      2045 & 0.870 & 1.28 & 14.46 & $1.48\pm0.03$ & ... \\
      2237 & 0.039 & 1.7  & 1.40  & $0.89\pm0.01$ & ... \\
    \end{tabular}
    \caption[width=\linewidth]{The most relevant lens properties for this work are listed \citep[for an expanded version of this table see][]{2011ApJ...740...97L}. References can be found in \S\ref{sec:data}. The columns are from left to right: the lens redshift $z_L$; the redshift of the source $z_S$; the maximum angular separation between two images $\Delta\theta$; the ratio of the radius of the outermost image $R_L$ and the effective Petrosian half-light radius of the lens $R_e$; and information on the environment of the lens. C and G denote a known cluster or group environment, respecitvely. The number in the brackets is the number of confirmed members. Ellipsis indicate lens galaxies with no group members identified so far.}
    \label{tab:lensproperties}
  \end{center}
\end{table}

\begin{table}
  \begin{center}
    \begin{tabular}{l l l}
      Lens & Time delays [days] & Additional priors \\ \hline
      0047 & ... & ...\\
      0414 & ... & ...\\
      0712 & ... & ...\\
      0911 & $\Delta t_{BA}=146^{+4}_{-4}$ & Symm\\
      0957 & $\Delta t_{BA}=416.5^{+1}_{-1}$ & Symm \\
      1115 & $\Delta t_{BA}=12.0^{+2}_{-2}$ & ...\\
           & $\Delta t_{DC}=4.4^{+3.2}_{-2.4}$ & \\
      1422 & ... & Symm \\
      1608 & $\Delta t_{BA}=31.5^{+2}_{-1}$ & ...\\
           & $\Delta t_{CA}=36.0^{+1.5}_{-1.5}$ & \\
           & $\Delta t_{DA}=77.0^{+2}_{-1}$ & \\
      2016 & ... & PixRad(16) \\
      2045 & ... & $\gamma(0.1)$, Symm \\
      2237 & ... & ...\\
    \end{tabular}
    \caption[width=\linewidth]{Additional priors to the base set described in \S\ref{sec:glass} are listed. $\Delta t_{BA}$ denotes the difference in arrival time $t_{B}-t_{A}$ between light from the images B and A. The images are defined in Table~\ref{tab:lenspositions}. $\gamma(\cdot)$ denotes a prior on the maximum allowed external shear $|\gamma|$. `Symm' denotes that a point-symmetry prior is employed. `PixRad(16)' indicates that the pixel radius was set to 16 pixels.}
    \label{tab:lenspriors}
  \end{center}
\end{table}


\section{Results}\label{sec:results}
\begin{figure*}
  \centering
  \includegraphics[width=.8\linewidth]{Figures/wedges_shears.pdf}
  \caption[width=\linewidth]{The reconstructed lenses are displayed according to the reconstructed total shapes of the dark matter halo $e_{dm}$ and the stellar component $e_{*}$. The wedges show the alignment of the semi-major axes of the components. In case of alignment of the semi-major axes (Aligned), the wedges are vertical. In case of alignment of the semi-major with the semi-minor axes (Anti-Aligned), the wedges are horizontal. The opening angle displays the statistical error in the shape estimate of the dark matter halo. Both the errors in the alignment and the ellipticity of the dark matter distribution denote the ranges $68\%$ of all the mass models lie within. The lenses are additionally color-coded in terms of the required external shear in modelling the lens.}
  \label{fig:wedgesall}
\end{figure*}

\begin{figure*}
  \centering
  \includegraphics[width=.8\linewidth]{Figures/wedges.pdf}
  \caption[width=\linewidth]{Similar to Figure~\ref{fig:wedgesall}. In each of the subfigures however we only consider pixels within 1, 2, 3, 4, or 5 $R_e$. If the outermost image of a system lies within one of the limiting radii, the system is kept in this panel, but dropped from subsequent panels.}
  \label{fig:wedgesradii}
\end{figure*}

We have applied \Glass\ to the data set described in \S\ref{sec:data} consisting of 11 lenses. The reconstructed stellar and dark mass maps for each individual lens are given in the Appendix.

Figure~\ref{fig:wedgesall} shows our results for the shapes (see \S\ref{sec:shapemethod}) and alignments of the dark matter and stellar mass distributions. In this Figure, the full distribution up to one pixel past the last-image radius $R_{L}$ is probed. There are several key points to note. Firstly, the reconstructed dark matter distribution of all lenses considered here is rounder than the stellar distribution. Secondly, most lenses lie in the bottom-left corner: the dark matter distributions have typically low eccentricities, while the stellar distributions can be rather elliptical. Thirdly, 3 out of the 11 systems, {\it0712}, {\it2016}, and {\it1608}, stand out due to their larger stellar ellipticity ($e_* > 0.4$). At the same time, they show a very strong alignment between their stellar and dark matter distributions. One of these lenses ({\it1608}) is an known merger of an early-type galaxy with a probable late-type galaxy \citep{2003ApJ...584..100S}. We suggest that the other two systems, {\it0712} and {\it2016}, may also be post-merger systems, explaining their similarities with {\it1608}.

Fourthly, notice that all dark matter and stellar distributions with weak external shear (see the contour bar that displays the shear $\gamma$) are aligned apart from {\it0047} which is almost spherical ($e_{dm}\sim0.05$). By contrast, systems requiring a larger external shear ({\it0414}, {\it0911}, and {\it1422}) can display a sizeable misalignment between the distributions. The latter two systems are located in dense environments (see Table~\ref{tab:lensproperties}) that are likely responsible for these misalignments. As described furthermore in \S\ref{sec:data}, {\it1422} is a challenging system to model the stellar distribution. {\it0414} has a perturbed along the line of sight contributing to the lensing effect \citep{2011MNRAS.413L..86C}. Besides these three lens systems, one stands out from the rest, however. In {\it1115}, the semi-major axis of the dark matter seems to be aligned with the semi-minor axis of the stellar component; it is the most strongly misaligned (anti-aligned) lens system of all lenses studied here. This may be explained by its external shear ($\gamma \sim 0.1$) emanating from its group environment (see Table \ref{tab:lensproperties}).

We performed tests for the robustness of our shape estimate and its dependence, in particular, on the shear prior. Only two systems, {\it1422} and {\it2045}, reacted to a stricter prior on the external shear $|\gamma|$. {\it1422} requires a rather large external shear ($|\gamma|\sim0.25$). Restricting the range of the allowed external shear values did not affect the misalignment of the dark matter and stellar mass distributions. It did, however, increase the ellipticity of the reconstructed mass distribution: we observed a trade-off between the galaxy's ellipticity and the magnitude of the external shear as might be expected. {\it2045} on the other hand is particularly interesting. Allowing the shear to roam free as for the other lenses, favours an anomalously high shear ($\gamma > 0.5$), with strong misalignment. However, such models also produce spurious extra images along the arc. We find that limiting the maximum shear $|\gamma|$ to 0.1 removes the extra images, though the lens then pushes on its maximum shear prior. We find that the mass map is robust against a stronger shear prior, and does not affect the alignment of the distributions. It produces well-aligned dark matter and stellar distributions. We believe that a strong shear prior for this particular lens is justified by the appearance of spurious extra images if higher shear is allowed.

Figure~\ref{fig:wedgesradii} is similar to Figure~\ref{fig:wedgesall} with the difference that shapes are now averaged within different multiples of $R_e$. There appears to be a trend that the ellipticity of the stellar distribution increases with increasing radius of enclosure. There is potentially a similar trend for the dark matter distribution, although the samples are also consistent with a constant ellipticity. {\it0957}, and to some degree {\it1422}, display, however, some signs of isophotal twist complicating the interpretation (see also Appendix).


\section{Conclusions}\label{sec:conclusions}
We have measured the shape and alignment of the stellar and dark matter distributions in 11 lens systems using a new non-parametric lens tool, \Glass. We focussed on lenses that have either time delay data or stellar mass maps that contribute significantly to the central potential to determine the projected shape of their dark matter haloes \citep{2014arXiv1401.7990C}. We measured the shape and alignment using the eigenvalues $\lambda_i$ and eigenvectors of the 2D moment of inertia tensor of the stellar and dark matter distributions, defining a {\it shape parameter} $e = 1 - \lambda_{2}/\lambda_{1}$ ($\lambda_{1} \geq \lambda_{2}$; see \S\ref{sec:shapemethod}). We averaged $e$ over the range 1-5$R_e$, where $R_e$ is the effective radius of the light profile.

Our key results are as follows:

\begin{itemize}
\item We found that the dark matter haloes are rounder than the light distribution over the range $1R_e < R < 5R_e$. As we average over larger radii, the lenses, except the ones with a large stellar ellipticity, become increasingly elliptical in their stellar distributions. The same trend can furthermore be seen for the dark matter, although the data are also consistent with no change in shape. The dark matter haloes are never more elliptical than $e_{dm} = 0.2$, while their stars can extend to $e_* > 0.3$.

\item Three systems have a high stellar ellipticity ($e_* > 0.45$) and correspondingly high alignment between light and dark. One of these -- {\it1608} -- is a known merging pair; we suggest that the other two ({\it0712} and {\it2016}) may also be recent post-merger systems. 

\item Galaxies with high dark matter ellipticity and weak external shear show strong alignment; those with strong shear ($\gamma \gtrsim 0.1$) can be highly misaligned. This is reassuring since isolated misaligned galaxies are expected to be unstable.
\end{itemize}

Our results provide a new constraint on galaxy formation models that must explain the origin of low eccentricity dark matter haloes (not expected in pure dark-matter-only simulations), and highly misaligned lens systems. A possible remedy is the fact that we constrain 2D projections of the lensing mass and the light distribution integrated along the line of sight. The misalignments we identified also present a new challenge for those alternative gravity theories in which the light is the only source of gravity in a galaxy, where light and dark must necessarily be highly correlated.


\section{Acknowledgements}\label{sec:acknowledgements}
We thank Simon Birrer for useful discussion. JIR would like to acknowledge support from SNF grant PP00P2\_128540/1. DL's research is part of the project GLENCO, funded under the European Seventh Framework Programme, Ideas, Grant Agreement n. 259349.


\bibliographystyle{mn2e}
\bibliography{paper}


\appendix
\section{Reconstructed Lenses}\label{sec:reconstructions}
\begin{table}
  \begin{center}
    \begin{tabular}{l r r r r}
      Lens & A [''] & B [''] & C [''] & D [''] \\ \hline
      0047 & 1.270 & -0.630 & 0.520 & -0.730 \\
           & 0.105 & -0.995 & -1.045 & 0.705 \\
      0414 & -0.472 & -1.061 & -1.1947 & 0.885 \\
           & 1.277 & -0.661 & -0.255 & -0.361 \\
      0712 & -0.013 & 0.795 & 0.747 & -0.391 \\
           & -0.804 & -0.156 & -0.292 & 0.307 \\
      0911 & 2.226 & -0.968 & -0.709 & -0.696 \\
           & 0.278 & -0.105 & -0.507 & 0.439 \\
      0957$^{a}$ & 1.408 & 0.182 & 2.860 & -1.540 \\
           & 5.034 & -1.018 & 3.470 & -0.050 \\
      1115 & 0.355 & -0.909 & -1.093 & 0.717 \\
           & 1.322 & -0.714 & -0.260 & -0.627 \\
      1422 & 1.079 & 0.357 & 0.742 & -0.205 \\
           & -0.095 & 0.973 & 0.656 & -0.147 \\
      1608 & -1.300 & -0.560 & -1.310 & 0.570 \\
           & -0.800 & 1.160 & 0.700 & -0.080 \\
      2016 & -1.735 & 0.335 & 0.437 & 1.268 \\
           & 1.778 & -1.450 & -1.435 & 0.276 \\
      2045 & 1.121 & 1.409 & 1.255 & -0.507 \\
           & 0.824 & 0.035 & 0.576 & -0.183 \\
      2237 & 0.598 & -0.075 & 0.791 & -0.710 \\
           & 0.758 & -0.939 & -0.411 & 0.271 \\
    \end{tabular}
    \caption[width=\linewidth]{The image positions of all the systems are listed in order of arrival time (image A has the shortest arrival time, image D the longest). The positions are relative to the centre of the lens galaxy. The first row contains the RA coordinate, the second the DEC coordinate of the image. \newline $^{a}$ The source lensed by {\it0957} has a second component. Images A and B are a double associated with the main galaxy component, images C and D with the lensed substructure.}
    \label{tab:lenspositions}
  \end{center}
\end{table}

In this Appendix, we show the results of our lens modelling for each individual lens (Figures~\ref{fig:lensreconstruction1}-\ref{fig:lensreconstruction3}). The panels show, from left to right: the arrival time surface; the surface mass density of the dark matter; and the surface mass density of the stars. The solid lines mark the eigenvalues and eigenvectors of the 2D moment of inertia tensor in each case; the dotted lines denote the ranges 68\% of all the eigenvectors of the dark matter distribution corresponding to an individual model lie within. Figure \ref{fig:wedgesall} is constructed from the ratio of largest to smallest eigenvalue in each case (to measure the shape parameter $e$), and the angle between the dark matter and stellar major axes. Note that the angular scale is always the same for the dark matter and stellar maps, but varies between the different lenses as marked on the Figure axes.

We note in Figure~\ref{fig:lensreconstruction2}, specifically for {\it0957} and to some degree also {\it1422}, twisting isophotes \citep[e.g.][]{1978ComAp...8...27B}.

\begin{figure*}
  \centering
  \includegraphics[width=.8\linewidth]{Figures/AllLenses31.pdf}
  \caption[width=.65\linewidth]{The results of our lens modelling for each individual lens. The panels show, from left to right: the arrival time surface (the images are marked by the green circles); the surface mass density of the dark matter; and the surface mass density of the stars. The solid lines mark the eigenvalues and eigenvectors of the 2D moment of inertia tensor in each case; the dotted lines the 68\% confidence interval of these for the dark matter map.}
  \label{fig:lensreconstruction1}
\end{figure*}

\begin{figure*}
  \centering
  \includegraphics[width=.8\linewidth]{Figures/AllLenses32.pdf}
  \caption[width=.65\linewidth]{The results of our lens modelling for each individual lens. Lines and symbols are as in Figure \ref{fig:lensreconstruction1}.}
  \label{fig:lensreconstruction2}
\end{figure*}

\begin{figure*}
  \centering
  \includegraphics[width=.8\linewidth]{Figures/AllLenses33.pdf}
  \caption[width=.65\linewidth]{The results of our lens modelling for each individual lens. Lines and symbols are as in Figure \ref{fig:lensreconstruction1}.}
  \label{fig:lensreconstruction3}
\end{figure*}

\end{document}