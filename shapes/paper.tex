\documentclass[useAMS,usenatbib]{mn2e}
\usepackage{myaasmacros}
\usepackage{graphicx}
\usepackage{ulem}
\usepackage{amsmath}
\usepackage{multirow}

% Some definitions of things I always use here:
\def\ltsima{$\; \buildrel < \over \sim \;$}
\def\simlt{\lower.5ex\hbox{\ltsima}}   
\def\gtsima{$\; \buildrel > \over \sim \;$}
\def\simgt{\lower.5ex\hbox{\gtsima}}

%Code definition:
\def\Glass{{\sc Glass}}
\def\PixeLens{{\sc PixeLens}}

\title[Light versus Dark in Strong Lensing Galaxies]{Light versus Dark in Strong  Lensing Galaxies: Dark matter halos are rounder than their stars}

\author[Bruderer]{Claudio Bruderer$^{1}$\thanks{E-mail: claudio.bruderer@phys.ethz.ch}, J. I. Read$^{1,2}$, P. Saha$^{3}$, J. Coles$^{4}$\\
$^{1}$Institute for Astronomy, Department of Physics, ETH Z\"urich, Wolfgang-Pauli-Strasse 27, CH-8093 Z\"urich, Switzerland\\
$^{2}$Department of Physics, University of Surrey, Guildford, GU2 7XH, UK\\
$^{3}$Institute for Theoretical Physics, University of Z\"urich, Winterthurerstrasse 190, 8057 Z\"urich, Switzerland\\
$^{4}$Exascale Research Computing Lab, Campus Teratec, 2 Rue de la Piquetterie, 91680 Bruyeres-le-Chatel, France
}

\begin{document}

\maketitle

\begin{abstract}
We measure the shape and alignment of stars and dark matter in 11 strong
lensing galaxies and find that in all cases the dark matter halos are rounder
than the light distribution over the range $R_e < R < 5R_e$.  Averaging over
larger radii, the lenses become increasingly elliptical both in their dark
matter and stellar components, as defined using the ratio $s$ of the largest to
smallest eigenvalues of the 2D moment of inertia tensor of the projected
lensing mass.  While dark matter halos are never more elliptical than $s_{dm} =
1.15$, their stars can extend to $s_* > 1.4$ Three systems, in particular, have
very high stellar ellipticity ($s_* > 1.6$) and correspondingly high alignment
between light and dark. One of these -- B1608 -- is a known merging pair; we
suggest that the other two (B0712 and B2016) may also be recent post-merger
systems.  Galaxies with high dark matter ellipticity and weak external shear
show very strong alignment between light and dark; those with strong shear
($\gamma \simgt 0.1$) can be very highly misaligned. This is reassuring since
isolated misaligned galaxies are expected to be unstable. 

Our results provide a new constraint on galaxy formation models that must
explain the origin of very round dark matter halos and highly misaligned
systems. Such misalignments also present a new challenge for alternative
gravity theories in which the light and dark must necessarily be highly
correlated.

\end{abstract}

\begin{keywords}
Gravitational lensing: strong --- galaxies: structure
\end{keywords}


\section{Introduction}\label{sec:introduction}

The ellipticity and shape of stars relative to their host dark matter halo encodes information both about our cosmological model ($\Lambda$CDM) and galaxy formation \citep[e.g.][]{1994ApJ...431..617D,2001ApJ...551..294I,2004ApJ...611L..73K,2007MNRAS.378...55M,2007arXiv0707.0737D,2012MNRAS.424L..16L,2014JPhG...41f3101R}. `Dark-matter-only' (DMO) simulations in $\Lambda$CDM predict dark matter halos that are triaxial \citep{1991ApJ...378..496D,1992ApJ...399..405W,1996ApJ...462..563N,2002ApJ...574..538J}, with mean `shape parameter' $\langle q \rangle = (b+c)/2a \sim 0.8$ (where $a > b > c$ are the long, intermediate and short axes of the figure; \citealt{2007MNRAS.378...55M}). This corresponds to a typically {\it prolate} halo. However, including `baryons' (stars and gas) in the models produces halos that are significantly rounder and -- at least for disc galaxies -- well-aligned with the light distribution \citep{1991ApJ...377..365K,1994ApJ...431..617D,2007arXiv0707.0737D}. Halo shapes and alignments also constrain alternative gravity models \citep{2001MNRAS.327..552M,2004ApJ...610L..97H,2005MNRAS.361..971R,2012PhRvD..86h3507F,2013MNRAS.434.2971D}. If the visible light is the only source of gravity in a galaxy, then we expect the light and mass distribution to be highly correlated; if dark matter is present, however, such correlations can, at least in principle, be broken.

Strong lensing provides a unique probe of the alignment and shape of the total mass distribution in galaxies \citep[e.g.][]{1986ApJ...310..568B,1992grle.book.....S,1998ApJ...509..561K,2000ApJ...543..131K,2006ApJ...649..599K,2007AJ....134..668A,2008MNRAS.383..857F,2010ApJ...724..511A,2012MNRAS.424..104L}. For `red and dead' ellipticals that are largely devoid of gas, their baryonic content can be mapped through stellar population synthesis modelling of their light distribution \citep[e.g.][]{2005ApJ...623L...5F,2006ApJ...640..662T,2008MNRAS.383..857F}. This opens up the possibility of directly comparing the light and mass in strong lensing systems \citep{1998ApJ...509..561K,2008MNRAS.383..857F,2009ApJ...690..670T,2012A&A...538A..99S}. Previous work in the literature has found that the light and mass are well-aligned (though a mis-match of up to 10$^\circ$ is not uncommon; e.g. \citealt{2012A&A...538A..99S}). However, results on the ellipticity of light and mass agree less well, with \citet{2012A&A...538A..99S} finding a strong correlation and \citet{1998ApJ...509..561K} and \citet{2008MNRAS.383..857F} finding none. It it difficult, however, to compare the results between these different studies because they use different lens modelling techniques; different definitions of ellipticity; and different radii over which the shapes and alignments are probed. Furthermore, none to date have applied their methodology to mock data to determine the robustness of the results.

Weak lensing can also be used to probe the shape and alignment between light and dark, but only for `stacked' galaxies \citep{2000astro.ph..6281B,2000ApJ...538L.113N}. \citet{2004ApJ...606...67H} applied this idea to data from the Red-Sequence Cluster Survey (RCS) to measure the first weak lensing signal of halo flattening. They found that dark matter halos appear to be rounder then their light distributions, with some weak evidence for alignment. Both measurements are challenging, however, and more recent data appear to be at odds with this early work, favouring dark matter halos that are more elliptical than their stars \citep{2006MNRAS.370.1008M,2007ApJ...669...21P,2012A&A...545A..71V}.

Recently, we introduced a new non-parametric lens tool, \Glass\  \citep{2014arXiv1401.7990C}. Applying this to a large suite of mock data, we showed that mass and light can only reliably be disentangled in strong lensing systems if: i) there are at least four images; and ii) time delay data are available and/or the stellar mass dominates the potential over the region of the images. In this paper, we collate data of the above quality, compiling a sample of 11 strong lens galaxies. We apply \Glass\ to these lenses to non-parametrically measure the shape and alignment of the stars and {\it dark matter}  in these lensing galaxies, for the first time. This differs from previous works that have all compared the light distribution with the total mass, rather than the dark matter. Since the stars often dominate the central potential, the total mass naturally correlates with the light, potentially masking theoretically interesting results about the dark matter distribution. Our comparison between light and dark is made possible by the fact that \Glass\ uses the light distribution as a prior on the mass map, ensuring that the dark matter mass is always positive.

This paper is organised as follows. In \S\ref{sec:glass}, we briefly review the \Glass\ code. In \S\ref{sec:data}, we present our data compilation with references. In \S\ref{sec:results}, we present our results. Finally, in \S\ref{sec:conclusions} we discuss the implications of these results and we present our main conclusions.


\section{\Glass: A non-parametric lens tool}\label{sec:glass}

\Glass{} is a new `non-parametric' gravitationl lens modelling framework
\citep{2014arXiv1401.7990C} that is capable of generating a distribution of
mass models given a set of basic assumptions, or priors.  In this context,
non-parametric means that  many more parameters than data constraints are used
such that system of equations describing the lensing mass distribution is
under-constrained.  \Glass{} shares some aspects with an earlier code
\PixeLens{} \citep{Saha2004,2008ApJ...679...17C}, but has a completely
different code base. It key improvements upon \PixeLens{} which are relevant
for this paper include:

\begin{enumerate}
\item A new uniform sampling algorithm for high
    dimensional spaces \citep{2012MNRAS.425.3077L}. This allows for large
    ensembles of $>10,000$ models to be efficiently generated. 
%\item \Glass\ provides a modular framework that allows new priors to be added
%    and modified easily.
%\item The basis functions approximating a model can be easily changed (in this
%    paper, we assume pixels as in \PixeLens). 
\item With so many models in the final ensemble, we can afford to apply
    non-linear constraints (for example stellar kinematic data; or the removal of
    models with spurious extra images) to accept/reject models in a post-processing
    step.
\item The central region of the mass map can have a higher resolution to more
    efficiently capture steep models.
\item Stellar density can be used as an additional lower-bound constraint on the models. 
%\item Point or extended mass objects can be placed in the field.
\end{enumerate} 

%\Glass{} is described in detail in the code paper: \citet{2014arXiv1401.7990C}.
In this paper, the default priors used to model each lens are listed below.  Table
\ref{tab:lensmodelling} identifies lenses for which we have modified these defaults.
For each lens we assume:

\begin{enumerate} 
\item Observational priors (redshifts, image positions relative to lens position).
\item Image parity is enforced.
\item A reconstructed stellar mass map.
\item The local density gradient is always pointing within 50$^{\circ}$ to the center and the slope is at most 0 everywhere. In other words, the density profile peaks in the center.
\item The fixed cosmology $\Omega_{m} = 0.28$, $\Omega_{\Lambda} = 0.72$, $\Omega_{k} = 0$, $H_{0}^{-1} = 13.7$ Gyr.
\item A flat prior on the magnitude of the external shear $|\gamma|$ on $[0,1)$.
\end{enumerate}

In cases where more information is available, the following priors are also used:

\begin{enumerate}
\item[(vii)] A flat prior on measured time delays within the $\pm 1\sigma$-interval.

\item[(viii)] External point masses to model satellite galaxies outside but
close to the Einstein radius.

\item[(ix)] Point symmetry $\kappa(x,y) = \kappa(-x,-y)$ where 
the surface mass profile can be assumed to be symmetric (3+1 quad
configurations and the double \textit{Q0957}).

\end{enumerate}

\subsection{Measuring the shape and alignment of a lens}\label{sec:shapemethod}

{\bf JR. WE NEED TO DEFINE OUR SHAPE PARAMETER AND ANALYSIS METHOD HERE.} 
{\bf JR. WE NEED TO ALSO DEFINE $R_e$ HERE AND EXPLAIN THAT WE AVERAGE SHAPE OVER 1-5$R_e$} 

\section{Data}\label{sec:data}

{\bf JR. CLAUDIO TO ADD DATA HERE.} 

\textbf{Content:}
\begin{itemize}
\item Describe data set (why this data set, special features of galaxies (environment: y/n/unknonwn, elliptical/disk))

\item {\bf JR. Claudio to add this bit?} 
\end{itemize}

\textit{Q0047-2808} (hereafter \textit{Q0047} or \textit{0047}) is a luminous early-type galaxy \citep{1996MNRAS.278..139W}. \cite{2011ApJ...726...84W} find a group of 9 members all of which they spectroscopically confirm.

\textit{MG0414+0534} (hereafter \textit{0414}) is a passively evolving early-type galaxy \citep{1999AJ....117.2034T}. \cite{1993AJ....105....1S} find a very close luminous satellite galaxy north-west of the lens. Due to the proximity of this satellite, we include it as a point mass in our analysis.

\textit{B0712+472} (hereafter \textit{0712}) is an early-type galaxy \citep{1998AJ....115..377F}. Another galaxy, 102'' to the south-east seems to be at a similar redshift \citep{2002AJ....123..627F}. There is also a group of 10 galaxies at a lower redshift.

\textit{RXJ0911+0551} (hereafter \textit{0911}) is an almost circular early-type galaxy \citep{2012A&A...538A..99S}. It has measured time delays \citep{2002ApJ...572L..11H}. It lies on the outskirts of a cluster \citep{2001ApJ...555....1M}. This cluster seems to be rather complex and especially it is not spherical. Possibly it is not dynamically relaxed, although X-ray emission can be detected. Analysis of this emission yields a temperature of 2.3 keV. There is also a satellite galaxy to the north-west direction \citep{2000ApJ...544L..35K}.

\textit{Q0957+561} (hereafter \textit{0957}) is a cD galaxy lying close to the center of a cluster with a high sprial galaxy-fraction (e.g. \cite{1992MNRAS.254P..27G,1994A&A...291..411A,1998ApJ...504..661C}). Due to the large image separation, large physical scales are probed. As listed in Table 3 of \cite{2000ApJ...542...74K}, the position angles to the center of the cluster of earlier lens reconstructions range between 51.8 deg and 67.8 deg measured north-through-east. These results are consistent with the center of the X-ray emission from the cluster \citep{1998ApJ...504..661C}. The lens has a measured time delay, e.g. \citep{2012A&A...540A.132S}. They however also find a three-day lag between the g- and r-bands, and the estimates do not agree at the 2$\sigma$-level. They argue that this effect can be accounted for by the presence of a substructure and chromatic dispersion. We find that the results do not change significantly for either estimate and choose therefore the g-band measurement.

\textit{PG1115+080} (hereafter \textit{1115}) is an early-type galaxy \citep{2005ApJ...626...51Y}. It has measured time delays. We use recent estimates of the time delays by \cite{2010MNRAS.406.2764T} which differ significantly from traditional values by e.g. \cite{1997ApJ...489...21B}. \cite{2006ApJ...641..169M} and \cite{2011ApJ...726...84W} analyzed the environment of the lens thouroughly. It is part of a small group of 13 members. \cite{2004ApJ...610..686G} detect also X-ray emission from the corresponding group, which yields a temperature of 0.8 keV.

\textit{B1422+231} (hereafter \textit{1422}) is an early-type galaxy \citep{1996ApJ...462L..53I}. Although there are measured time delays by \cite{2001MNRAS.326.1403P}, they are probably not to be trusted too much as they deviate significantly from theoretical expectations in \cite{2003AJ....126...29R}. An accurate measurement of the time delays would require time delay measurements on the scale of hours. \cite{2006ApJ...641..169M} find a group environment with 16 spectroscopically confirmed member galaxies. Using newer data, \cite{2011ApJ...726...84W} find an additional member. \cite{2004ApJ...610..686G} also detect X-ray emission from the corresponding group at a temperature of 1.0 keV.

\textit{B1608+656} (hereafter \textit{1608} or \textit{1608}) consists of two merging galaxies. The main galaxy is an early-type galaxy which is disrupted by a smaller, probably late-type galaxy \citep{2003ApJ...584..100S}. The system has measured time delays \citep{2002ApJ...581..823F}. The environment and the mass distribution along the line of sight have been analyzed by \cite{2006ApJ...642...30F}. They find a group with 8 resp. 9 members if the merging galaxies are counted as 2. No significant X-ray emission was detected from the surrounding group \citep{2005ApJ...625..633D}. Along the line of sight seem to be four other groups.

\textit{MG2016+112} (hereafter \textit{2016}) is a giant elliptical galaxy \citep{1984Sci...223...46L,1986AJ.....91..991S}. It is the farthest lens we consider in this sample. It lies in a cluster which consists of 69 probable galaxies \citep{2003MNRAS.344..337T}. The clusters shows a high density of galaxies close to the lens in a south-east direction.

\textit{B2045+265} (hereafter \textit{2045}) is probably an elliptical galaxy \citep{2007MNRAS.378..109M}. \cite{1999AJ....117..658F} initially classified the galaxy as a late-type Sa galaxy, the velocity dispersion however seems too high. As the source redshift is rather low, a large lens mass is needed. \cite{2007MNRAS.378..109M} therefore conclude that it is more likely an elliptical galaxy. To the west of the lens, \cite{1999AJ....117..658F} find that a group at a similar redshift as the lens. \cite{2007MNRAS.378..109M} also find evidence for a dwarf satellite galaxy. As the measurements are not conclusive, we chose to treat the lens as an isolated object.

\textit{Q2237+030} (hereafter \textit{2237}) is a barred spiral \citep{1988AJ.....95.1331Y}. With only a redshift of $z=0.04$, it is the closest lens of the sample. Due to the low redshift, the probed physical scales are quite small. No further inquiries into the environment were made.

Further information about the sample can be found in \cite{leier11phd}, \cite{2011ApJ...740...97L}, and \cite{2012A&A...538A..99S}.

\begin{table*}
  \begin{center}
    \begin{tabular}{l r r r r l}
      Lens    & \multicolumn{1}{c}{$z_{L}$} & \multicolumn{1}{c}{$z_{S}$} & \multicolumn{1}{c}{$\Delta\theta$ [kpc]} & \multicolumn{1}{c}{$R_{L}$/$R_{e}$} & Environment \\ \hline
      0047 & 0.485 & 3.60 & 12.82 & $1.45\pm0.04$ & G(9) \\
      0414 & 0.960 & 2.64 & 16.01 & $1.85\pm0.05$ & ... \\
      0712 & 0.410 & 1.34 & 6.82  & $1.15\pm0.03$ & ... \\
      0911 & 0.769 & 2.8  & 23.16 & $3.09\pm0.05$ & C \\
      0957 & 0.356 & 1.41 & 29.98 & $3.51\pm0.04$ & C \\
      1115 & 0.310 & 1.72 & 10.76 & $2.94\pm0.06$ & G(13) \\
      1422 & 0.337 & 3.62 & 6.02  & $4.49\pm0.06$ & G(17) \\
      1608 & 0.630 & 1.39 & 13.92 & $1.82\pm0.01$ & !CHECK! \\
      2016 & 1.010 & 3.3  & 26.22 & $6.12\pm0.14$ & C(69) \\
      2045 & 0.870 & 1.28 & 14.46 & $1.48\pm0.03$ & !SEE TEXT! \\
      2237 & 0.039 & 1.7  & 1.40  & $0.89\pm0.01$ & ... \\
    \end{tabular}
    \caption[width=\linewidth]{Table with lens properties \citep[see][]{2011ApJ...740...97L}. References in Section~\ref{sec:data} \textbf{expand}}
    \label{tab:lensproperties}
  \end{center}
\end{table*}

\begin{table*}
  \begin{center}
    \begin{tabular}{l r r r r l l}
      Lens    & \multicolumn{1}{c}{A ['']} & \multicolumn{1}{c}{B ['']} & \multicolumn{1}{c}{C ['']} & \multicolumn{1}{c}{D ['']} & Time delays [days] & Additional priors \\ \hline
      0047 & 1.270 & -0.630 & 0.520 & -0.730 & ... & \\
           & 0.105 & -0.995 & -1.045 & 0.705 & &  \\
      0414 & -0.472 & -1.061 & -1.1947 & 0.885 & ... & \\
           & 1.277 & -0.661 & -0.255 & -0.361 & &  \\
      0712 & -0.013 & 0.795 & 0.747 & -0.391 & ... & \\
           & -0.804 & -0.156 & -0.292 & 0.307 & &  \\
      0712 & -0.013 & 0.795 & 0.747 & -0.391 & ... & \\
           & -0.804 & -0.156 & -0.292 & 0.307 &  &  \\
      0957$^{a}$ & 1.408 & 0.182 & 2.860 & -1.540 & $\Delta t_{BA}=416.5^{+1}_{-1}$ & Symm \\
           & 5.034 & -1.018 & 3.470 & -0.050 &  &  \\
      1115 & 0.355 & -0.909 & -1.093 & 0.717 & $\Delta t_{BA}=12.0^{+2}_{-2} \ \Delta t_{DC}=4.4^{+3.2}_{-2.4}$ & \\
           & 1.322 & -0.714 & -0.260 & -0.627 & &  \\
      1422 & 1.079 & 0.357 & 0.742 & -0.205 & ... & Symm \\
           & -0.095 & 0.973 & 0.656 & -0.147 &  &  \\
      1608 & -1.300 & -0.560 & -1.310 & 0.570 & $\Delta t_{BA}=31.5^{+2}_{-1} \ \Delta t_{CA}=36.0^{+1.5}_{-1.5} \ \Delta t_{DA}=77.0^{+2}_{-1}$ & \\
           & -0.800 & 1.160 & 0.700 & -0.080 &  &  \\
      2016 & -1.735 & 0.335 & 0.437 & 1.268 & ... & \\
           & 1.778 & -1.450 & -1.435 & 0.276 &  &  \\
      2045 & 1.121 & 1.409 & 1.255 & -0.507 & ... & $\gamma(0.1)$, Symm \\
           & 0.824 & 0.035 & 0.576 & -0.183 &  &  \\
      2237 & 0.598 & -0.075 & 0.791 & -0.710 & ... & \\
              & 0.758 & -0.939 & -0.411 & 0.271 &  &  \\
    \end{tabular}
    \caption[width=\linewidth]{Table with lens properties relevant for modelling (positions, time delays) \newline $^{a}$ Images A and B are a pair, C and D the other \textbf{expand}}
    \label{tab:lensmodelling}
  \end{center}
\end{table*}


\section{Results}\label{sec:results}

{\bf JR. CLAUDIO OR JUSTIN TO WRITE RESULTS SECTION?} 
\textbf{Content:}
\begin{itemize}
\item Describe special features in reconstructed lenses
\item Show the wedges money plot
\item Discuss the results, especially:
\item 1. Dark matter halos seem quite round, stars not necessarily
\item 2. Both DM and stars more elliptical with increasing radius 
\item 3. DM halos with weak shear are aligned (apart from spherical systems that don't count); strong shear systems can be highly misaligned.
\item 4. No sensitivity to shear prior. Only one lens B2045 requires a shear prior to avoid spurious extra images; doesn't affect the shape measure though. 
\item 5. Merger systems stand out. 
\end{itemize}

\begin{figure}
  \centering
  \includegraphics[width=\linewidth]{Figures/wedges_shears.pdf}
  \caption[width=.65\linewidth]{}
  \label{fig:wedges}
\end{figure}

\begin{figure*}
  \centering
  \includegraphics[width=.8\linewidth]{Figures/wedges.pdf}
  \caption[width=.65\linewidth]{}
  \label{fig:wedgesall}
\end{figure*}


\section{Conclusion}\label{sec:conclusions}

We have measured the shape and alignment of stars and dark matter in 11 strong lensing galaxies using a new non-parametric lens tool, \Glass. We focussed on lenses that have either time delay data or stellar mass maps that contribute significantly to the central potential, since these have data quality good enough to determine the projected shape of their dark matter halos \citep{2014arXiv1401.7990C}. We measured the shape and alignment using the eigenvalues $\lambda$ and vectors of the 2D moment of inertia tensor of the stars and dark matter, defining a {\it shape parameter} $s = \lambda_{\rm max}/\lambda_{\rm min}$ (\S\ref{sec:shapemethod}). We averaged $s$ over the range 1-5$R_e$, where $R_e$ is the effective radius of the light profile.

Our key results are as follows: 

\begin{itemize}
\item We found that in all cases the dark matter halos are more elliptical than the light distribution over the range $R_e < R < 5R_e$. As we average over larger radii, the lenses become increasingly elliptical both in their dark matter and stars, but dark matter halos are never more elliptical than $s_{dm} = 1.15$, while their stars can extend to $s_* > 1.4$. 

\item Three systems have very high stellar ellipticity ($s_* > 1.6$) and correspondingly high alignment between light and dark. One of these -- B1608 -- is a known merging pair; we suggest that the other two (B0712 and B2016) may also be recent post-merger systems. 

\item Galaxies with high dark matter ellipticity and weak external shear show very strong alignment between light and dark; those with strong shear ($\gamma \simgt 0.1$) can be very highly misaligned. This is reassuring since isolated misaligned galaxies are expected to be unstable.
\end{itemize}

Our results provide a new constraint on galaxy formation models that must explain the origin of very round dark matter halos (not expected in pure dark matter only simulations), and highly misaligned systems. Such misalignments also present a new challenge for alternative gravity theories in which the light and dark must necessarily be highly correlated.

\section{Acknowledgements}\label{sec:acknowledgements}
We would like to thank Dominik Leier for useful discussions. JIR would like to acknowledge support from SNF grant PP00P2\_128540/1.

\bibliographystyle{mn2e}
\bibliography{paper}

\appendix
\section{Reconstructed Lenses}\label{sec:reconstructions}

In this appendix, we show the results of our lens modelling for each individual lens (Figures \ref{fig:lensreconstruction1}, \ref{fig:lensreconstruction2}, and \ref{fig:lensreconstruction3}). The panels show, from left to right: the arrival time surface; the surface mass density of the dark matter; and the surface mass density of the stars. The solid lines mark the eigenvalues and eigenvectors of the 2D moment of inertia tensor in each case; the dotted lines the 68\% confidence interval of these for the dark matter map. Figure \ref{fig:wedges} is constructed from the ratio of largest to smallest eigenvalue in each case (to measure the shape parameter $s$), and the angle between the dark matter and stellar major axes. Note that the angular scale is always the same for the dark matter and stellar maps, but varies between the different lenses as marked on the Figure axes. 

\begin{figure*}
  \centering
  \includegraphics[width=.8\linewidth]{Figures/AllLenses31.pdf}
  \caption[width=.65\linewidth]{The results of our lens modelling for each individual lens. The panels show, from left to right: the arrival time surface (the images are marked by the green circles); the surface mass density of the dark matter; and the surface mass density of the stars. The solid lines mark the eigenvalues and eigenvectors of the 2D moment of inertia tensor in each case; the dotted lines the 68\% confidence interval of these for the dark matter map.}
  \label{fig:lensreconstruction1}
\end{figure*}

\begin{figure*}
  \centering
  \includegraphics[width=.8\linewidth]{Figures/AllLenses32.pdf}
  \caption[width=.65\linewidth]{The results of our lens modelling for each individual lens. Lines and symbols are as in Figure \ref{fig:ensreconstruction1}.}
  \label{fig:lensreconstruction2}
\end{figure*}

\begin{figure*}
  \centering
  \includegraphics[width=.8\linewidth]{Figures/AllLenses33.pdf}
  \caption[width=.65\linewidth]{The results of our lens modelling for each individual lens. Lines and symbols are as in Figure \ref{fig:ensreconstruction1}.}
  \label{fig:lensreconstruction3}
\end{figure*}

\end{document}
